%
% (c) The OBFUSCATION-THROUGH-GRATUITOUS-PREPROCESSOR-ABUSE Project,
%     Glasgow University, 1990-1994
%

% TODO:
%
% o I (ADR) think it would be worth making the connection with CPS explicit.
%   Now that we have explicit activation records (on the stack), we can
%   explain the whole system in terms of CPS and tail calls --- with the
%   one requirement that we carefuly distinguish stack-allocated objects
%   from heap-allocated objects.

% \documentstyle[preprint]{acmconf}
\documentclass[11pt]{article}
\oddsidemargin 0.1 in       %   Note that \oddsidemargin = \evensidemargin
\evensidemargin 0.1 in
\marginparwidth 0.85in    %   Narrow margins require narrower marginal notes
\marginparsep 0 in 
\sloppy

%\usepackage{epsfig}
\usepackage{shortvrb}
\MakeShortVerb{\@}

%\newcommand{\note}[1]{{\em Note: #1}}
\newcommand{\note}[1]{{{\bf Note:}\sl #1}}
\newcommand{\ToDo}[1]{{{\bf ToDo:}\sl #1}}
\newcommand{\Arg}[1]{\mbox{${\tt arg}_{#1}$}}
\newcommand{\bottom}{\perp}

\newcommand{\secref}[1]{Section~\ref{sec:#1}}
\newcommand{\figref}[1]{Figure~\ref{fig:#1}}
\newcommand{\Section}[2]{\section{#1}\label{sec:#2}}
\newcommand{\Subsection}[2]{\subsection{#1}\label{sec:#2}}
\newcommand{\Subsubsection}[2]{\subsubsection{#1}\label{sec:#2}}

% DIMENSION OF TEXT:
\textheight 8.5 in
\textwidth 6.25 in

\topmargin 0 in
\headheight 0 in
\headsep .25 in


\setlength{\parskip}{0.15cm}
\setlength{\parsep}{0.15cm}
\setlength{\topsep}{0cm}	% Reduces space before and after verbatim,
				% which is implemented using trivlist 
\setlength{\parindent}{0cm}

\renewcommand{\textfraction}{0.2}
\renewcommand{\floatpagefraction}{0.7}

\begin{document}

\title{The STG runtime system (revised)}
\author{Simon Peyton Jones \\ Microsoft Research Ltd., Cambridge \and
Simon Marlow \\ Microsoft Research Ltd., Cambridge \and
Alastair Reid \\ Yale University} 

\maketitle

\tableofcontents
\newpage

\part{Introduction}
\Section{Overview}{overview}

This document describes the GHC/Hugs run-time system.  It serves as 
a Glasgow/Yale/Nottingham ``contract'' about what the RTS does.

\Subsection{New features compared to GHC 3.xx}{new-features}

\begin{itemize}
\item The RTS supports mixed compiled/interpreted execution, so
that a program can consist of a mixture of GHC-compiled and Hugs-interpreted
code.

\item The RTS supports concurrency by default.
This has some costs (eg we can't do hardware stack checks) but
reduces the number of different configurations we need to support.

\item CAFs are only retained if they are
reachable.  Since they are referred to by implicit references buried
in code, this means that the garbage collector must traverse the whole
accessible code tree.  This feature eliminates a whole class of painful
space leaks.

\item A running thread has only one stack, which contains a mixture of
pointers and non-pointers.  \secref{TSO} describes how we find out
which is which.  (GHC has used two stacks for some while.  Using one
stack instead of two reduces register pressure, reduces the size of
update frames, and eliminates ``stack-stubbing'' instructions.)

\item The ``return in registers'' return convention has been dropped
because it was complicated and doesn't work well on register-poor
architectures.  It has been partly replaced by unboxed tuples
(\secref{unboxed-tuples}) which allow the programmer to
explicitly state where results should be returned in registers (or on
the stack) instead of on the heap.

\item Exceptions are supported by the RTS.

\item Weak Pointers generalise the previously available Foreign Object
interface.

\item The garbage collector supports a number of new features,
including a dynamically resizable heap and multiple generations with
aging within a generation.

\end{itemize}

\Subsection{Wish list}{wish-list}

Here's a list of things we'd like to support in the future.
\begin{itemize}
\item Interrupts, speculative computation.

\item
The SM could tune the size of the allocation arena, the number of
generations, etc taking into account residency, GC rate and page fault
rate.

\item 
We could trigger a GC when all threads are blocked waiting for IO if
the allocation arena (or some of the generations) are nearly full.

\end{itemize}

\Subsection{Configuration}{configuration}

Some of the above features are expensive or less portable, so we
envision building a number of different configurations supporting
different subsets of the above features.

You can make the following choices:
\begin{itemize}
\item
Support for parallelism.  There are three mutually-exclusive choices.

\begin{description}
\item[@SEQUENTIAL@] Support for concurrency but not for parallelism.
\item[@GRANSIM@]    Concurrency support and simulated parallelism.
\item[@PARALLEL@]   Concurrency support and real parallelism.
\end{description}

\item @PROFILING@ adds cost-centre profiling.

\item @TICKY@ gathers internal statistics (often known as ``ticky-ticky'' code).

\item @DEBUG@ does internal consistency checks.

\item Persistence. (well, not yet).

\item
Which garbage collector to use.  At the moment we
only anticipate one, however.
\end{itemize}

\Subsection{Glossary}{glossary}

\ToDo{This terminology is not used consistently within the document.
If you find something which disagrees with this terminology, fix the
usage.}

In the type system, we have boxed and unboxed types.

\begin{itemize}

\item A \emph{pointed} type is one that contains $\bot$.  Variables with
pointed types are the only things which can be lazily evaluated.  In
the STG machine, this means that they are the only things that can be 
\emph{entered} or \emph{updated} and it requires that they be boxed.

\item An \emph{unpointed} type is one that does not contain $\bot$.
Variables with unpointed types are never delayed --- they are always
evaluated when they are constructed.  In the STG machine, this means
that they cannot be \emph{entered} or \emph{updated}.  Unpointed objects
may be boxed (like @Array#@) or unboxed (like @Int#@).

\end{itemize}

In the implementation, we have different kinds of objects:

\begin{itemize}

\item \emph{boxed} objects are heap objects used by the evaluators

\item \emph{unboxed} objects are not heap allocated

\item \emph{stack} objects are allocated on the stack

\item \emph{closures} are objects which can be \emph{entered}. 
They are always boxed and always have boxed types.
They may be in WHNF or they may be unevaluated.  

\item A \emph{thunk} is a (representation of) a value of a \emph{pointed}
type which is \emph{not} in WHNF.

\item A \emph{value} is an object in WHNF.  It can be pointed or unpointed.

\end{itemize}



At the hardware level, we have \emph{word}s and \emph{pointer}s.

\begin{itemize}

\item A \emph{word} is (at least) 32 bits and can hold either a signed
or an unsigned int.

\item A \emph{pointer} is (at least) 32 bits and big enough to hold a
function pointer or a data pointer.  

\end{itemize}

Occasionally, a field of a data structure must hold either a word or a
pointer.  In such circumstances, it is \emph{not safe} to assume that
words and pointers are the same size.  \ToDo{GHC currently makes words
the same size as pointers to reduce complexity in the code
generator/RTS.  It would be useful to relax this restriction, and have
eg. 32-bit Ints on a 64-bit machine.}

% should define terms like SRT, CAF, PAP, etc. here?  --KSW 1999-03

\subsection{Subtle Dependencies}

Some decisions have very subtle consequences which should be written
down in case we want to change our minds.  

\begin{itemize}

\item

If the garbage collector is allowed to shrink the stack of a thread,
we cannot omit the stack check in return continuations
(\secref{heap-and-stack-checks}).

\item

When we return to the scheduler, the top object on the stack is a closure.
The scheduler restarts the thread by entering the closure.

\secref{hugs-return-convention} discusses how Hugs returns an
unboxed value to GHC and how GHC returns an unboxed value to Hugs.

\item 

When we return to the scheduler, we need a few empty words on the stack
to store a closure to reenter.  \secref{heap-and-stack-checks}
discusses who does the stack check and how much space they need.

\item

Heap objects never contain slop --- this is required if we want to
support mostly-copying garbage collection.

This is a big problem when updating since the updatee is usually
bigger than an indirection object.  The fix is to overwrite the end of
the updatee with ``slop objects'' (described in
\secref{slop-objects}).  This is hard to arrange if we do
\emph{lazy} blackholing (\secref{lazy-black-holing}) so we
currently plan to blackhole an object when we push the update frame.

% Idea: have specialised update code for various common sizes of
% updatee, the update frame hence encodes the length of the object.
% Specialised indirections will also encode the length of the object.  A
% generic version of the update code will overwrite the slop with a slop
% object.  We can do the same thing for blackhole objects, or just have
% a generic version that is the same size as an indirection and
% overwrite the slop with a slop object when blackholing.  So: does this
% avoid the need to do eager black holing?

\item

Info tables for constructors contain enough information to decide which
return convention they use.  This allows Hugs to use a single piece of
entry code for all constructors and insulates Hugs from changes in the
choice of return convention.

\end{itemize}

\Section{Source Language}{source-language}

\Subsection{Explicit Allocation}{explicit-allocation}

As in the original STG machine, (almost) all heap allocation is caused
by executing a let(rec).  Since we no longer support the return in
registers convention for data constructors, constructors now cause heap
allocation and so they should be let-bound.

For example, we now write
\begin{verbatim}
> cons = \ x xs -> let r = (:) x xs in r
@
instead of
\begin{verbatim}
> cons = \ x xs -> (:) x xs
\end{verbatim}

\note{For historical reasons, GHC doesn't use this syntax --- but it should.}

\Subsection{Unboxed tuples}{unboxed-tuples}

Functions can take multiple arguments as easily as they can take one
argument: there's no cost for adding another argument.  But functions
can only return one result: the cost of adding a second ``result'' is
that the function must construct a tuple of ``results'' on the heap.
The asymmetry is rather galling and can make certain programming
styles quite expensive.  For example, consider a simple state
monad:
\begin{verbatim}
> type S a     = State -> (a,State)
> bindS m k s0 = case m s0 of { (a,s1) -> k a s1 }
> returnS a s  = (a,s)
> getS s       = (s,s)
> setS s _     = ((),s)
\end{verbatim}
Here, every use of @returnS@, @getS@ or @setS@ constructs a new tuple
in the heap which is instantly taken apart (and becomes garbage) by
the case analysis in @bind@.  Even a short program using the state monad
will construct a lot of these temporary tuples.

Unboxed tuples provide a way for the programmer to indicate that they
do not expect a tuple to be shared and that they do not expect it to
be allocated in the heap.  Syntactically, unboxed tuples are just like
single constructor datatypes except for the annotation @unboxed@.
\begin{verbatim}
> data unboxed AAndState# a = AnS a State
> type S a = State -> AAndState# a
> bindS m k s0 = case m s0 of { AnS a s1 -> k a s1 }
> returnS a s  = AnS a s
> getS s       = AnS s s
> setS s _     = AnS () s
\end{verbatim}
Semantically, unboxed tuples are just unlifted tuples and are subject
to the same restrictions as other unpointed types.

Operationally, unboxed tuples are never built on the heap.  When
an unboxed tuple is returned, it is returned in multiple registers
or multiple stack slots.  At first sight, this seems a little strange
but it's no different from passing double precision floats in two
registers.

Notes:
\begin{itemize}
\item
Unboxed tuples can only have one constructor and that
thunks never have unboxed types --- so we'll never try to update an
unboxed constructor.  The restriction to a single constructor is
largely to avoid garbage collection complications.

\item
The core syntax does not allow variables to be bound to
unboxed tuples (ie in default case alternatives or as function arguments)
and does not allow unboxed tuples to be fields of other constructors.
However, there's no harm in allowing it in the source syntax as a
convenient, but easily removed, syntactic sugar.

\item
The compiler generates a closure of the form
\begin{verbatim}
> c = \ x y z -> C x y z
\end{verbatim}
for every constructor (whether boxed or unboxed).  

This closure is normally used during desugaring to ensure that
constructors are saturated and to apply any strictness annotations.
They are also used when returning unboxed constructors to the machine
code evaluator from the bytecode evaluator and when a heap check fails
in a return continuation for an unboxed-tuple scrutinee.

\end{itemize}

\Subsection{STG Syntax}{stg-syntax}


\ToDo{Insert STG syntax with appropriate changes.}


%%%%%%%%%%%%%%%%%%%%%%%%%%%%%%%%%%%%%%%%%%%%%%%%%%%%%%%%%%%%%%%%
\part{System Overview}
%%%%%%%%%%%%%%%%%%%%%%%%%%%%%%%%%%%%%%%%%%%%%%%%%%%%%%%%%%%%%%%%

This part is concerned with defining the external interfaces of the
major components of the system; the next part is concerned with their
inner workings.

The major components of the system are:
\begin{itemize}

\item 

The evaluators (\secref{sm-overview}) are responsible for
evaluating heap objects.  The system supports two evaluators: the
machine code evaluator; and the bytecode evaluator.

\item 

The scheduler (\secref{scheduler-overview}) acts as the
coordinator for the whole system.  It is responsible for switching
between evaluators, switching between threads, garbage collection,
communication between multiple processors, etc.

\item 

The storage manager (\secref{evaluators-overview}) is
responsible for allocating blocks of contiguous memory and for garbage
collection.

\item 

The loader (\secref{loader-overview}) is responsible for
loading machine code and bytecode files from the file system and for
resolving references between separately compiled modules.

\item 

The compilers (\secref{compilers-overview}) generate machine
code and bytecode files which can be loaded by the loader.

\end{itemize}

\ToDo{Insert diagram showing all components underneath the scheduler
and communicating only with the scheduler}


\Section{The Evaluators}{evaluators-overview}

There are two evaluators: a machine code evaluator and a bytecode
evaluator.  The evaluators task is to evaluate code within a thread
until one of the following happens:

\begin{itemize}
\item heap overflow
\item stack overflow
\item it is preempted
\item it blocks in one of the concurrency primitives
\item it performs a safe ccall
\item it needs to switch to the other evaluator.
\end{itemize}

The evaluators expect to find a closure on top of the thread's stack
and terminate with a closure on top of the thread's stack.

\Subsection{Evaluation Model}{evaluation-model}

Whilst the evaluators differ internally, they share a common
evaluation model and many object representations.

\Subsubsection{Heap objects}{heap-objects-overview}

The choice of heap and stack objects used by the evaluators is tightly
bound to the evaluation model.  This section provides an overview of
the most important heap and stack objects; further details are given
later.

All heap objects look like this:

\begin{center}
\begin{tabular}{|l|l|l|l|}\hline
\emph{Header} & \emph{Payload} \\ \hline
\end{tabular}
\end{center}

The headers vary between different kinds of object but they all start
with a pointer to a pair consisting of an \emph{info table} and some
\emph{entry code}.  The info table is used both by the evaluators and
by the storage manager and contains a @type@ field which identifies
which kind of heap object uses it and determines the interpretation of
the payload and of the other fields of the info table.  The entry code
is some machine code used by the machine code evaluator to evaluate
closures and raises an error for other kinds of objects.

The major kinds of heap object used are as follows.  (For simplicity,
this description omits certain optimisations and extra fields required
by the garbage collector.)

\begin{description}

\item[Constructors] are used to represent data constructors.  Their
payload consists of the fields of the constructor; the tag of the
constructor is stored in the info table.

\begin{center}
\begin{tabular}{|l|l|l|l|}\hline
@CONSTR@ & \emph{Fields} \\ \hline
\end{tabular}
\end{center}

\item[Primitive objects] are used to represent objects with unlifted
types which are too large to fit in a register (or stack slot) or for
which sharing must be preserved.  Primitive objects include large
objects such as multiple precision integers and immutable arrays and
mutable objects such as mutable arrays, mutable variables, MVar's,
IVar's and foreign object pointers.  Since primitive objects are not
lifted, they cannot be entered.  Their payload varies according to the
kind of object.

\item[Function closures] are used to represent functions.  Their
payload (if any) consists of the free variables of the function.

\begin{center}
\begin{tabular}{|l|l|l|l|}\hline
@FUN@ & \emph{Free Variables} \\ \hline
\end{tabular}
\end{center}

Function closures are only generated by the machine code compiler.

\item[Thunks] are used to represent unevaluated expressions which will
be updated with their result.  Their payload (if any) consists of the
free variables of the function.  The entry code for a thunk starts by
pushing an \emph{update frame} onto the stack.  When evaluation of the
thunk completes, the update frame will cause the thunk to be
overwritten again with an \emph{indirection} to the result of the
thunk, which is always a constructor or a partial application.

\begin{center}
\begin{tabular}{|l|l|l|l|}\hline
@THUNK@ & \emph{Free Variables} \\ \hline
\end{tabular}
\end{center}

Thunks are only generated by the machine code evaluator.

\item[Byte-code Objects (@BCO@s)] are generated by the bytecode
compiler.  In conjunction with \emph{updatable applications} and
\emph{non-updatable applications} they are used to represent
functions, unevaluated expressions and return addresses.

\begin{center}
\begin{tabular}{|l|l|l|l|}\hline
@BCO@ & \emph{Constant Pool} & \emph{Bytecodes} \\ \hline
\end{tabular}
\end{center}

\item[Non-updatable (Partial) Applications] are used to represent the
application of a function to an insufficient number of arguments.
Their payload consists of the function and the arguments received so far.

\begin{center}
\begin{tabular}{|l|l|l|l|}\hline
@PAP@ & \emph{Function Closure} & \emph{Arguments} \\ \hline
\end{tabular}
\end{center}

@PAP@s are used when a function is applied to too few arguments and by
code generated by the lambda-lifting phase of the bytecode compiler.

\item[Updatable Applications] are used to represent the application of
a function to a sufficient number of arguments.  Their payload
consists of the function and its arguments.  

Updateable applications are like thunks: on entering an updateable
application, the evaluators push an \emph{update frame} onto the stack
and overwrite the application with a \emph{black hole}; when
evaluation completes, the evaluators overwrite the application with an
\emph{indirection} to the result of the application.

\begin{center}
\begin{tabular}{|l|l|l|l|}\hline
@AP@ & \emph{Function Closure} & \emph{Arguments} \\ \hline
\end{tabular}
\end{center}

@AP@s are only generated by the bytecode compiler.

\item[Black holes] are used to mark updateable closures which are
currently being evaluated.  ``Black holing'' an object cures a
potential space leak and detects certain classes of infinite loops.
More imporantly, black holes act as synchronisation objects between
separate threads: if a second thread tries to enter an updateable
closure which is already being evaluated, the second thread is added
to a list of blocked threads and the thread is suspended.

When evaluation of the black-holed closure completes, the black hole
is overwritten with an indirection to the result of the closure and
any blocked threads are restored to the runnable queue.

Closures are overwritten by black-holes during a ``lazy black-holing''
phase which runs on each thread when it returns to the scheduler.
\ToDo{section describing lazy black-holing}.

\begin{center}
\begin{tabular}{|l|l|l|l|}\hline
@BLACKHOLE@ & \emph{Blocked threads} \\ \hline
\end{tabular}
\end{center}

\ToDo{In a single threaded system, it's trivial to detect infinite
loops: reentering a BLACKHOLE is always an error.  How easy is it in a
multi-threaded system?}

\item[Indirections] are used to update an unevaluated closure with its
(usually fully evaluated) result in situations where it isn't possible
to perform an update in place.  (In the current system, we always
update with an indirection to avoid duplicating the result when doing
an update in place.)

\begin{center}
\begin{tabular}{|l|l|l|l|}\hline
@IND@ & \emph{Closure} \\ \hline
\end{tabular}
\end{center}

Indirections needn't always point to a closure in WHNF.  They can
point to a chain of indirections which point to an evaluated closure.

\item[Thread State Objects (@TSO@s)] represent Haskell threads.  Their
payload consists of some per-thread information such as the Thread ID
and the status of the thread (runnable, blocked etc.), and the
thread's stack.  See @TSO.h@ for the full story.  @TSO@s may be
resized by the scheduler if its stack is too small or too large.

The thread stack grows downwards from higher to lower addresses.

\begin{center}
\begin{tabular}{|l|l|l|l|}\hline
@TSO@ & \emph{Thread info} & \emph{Stack} \\ \hline
\end{tabular}
\end{center}

\end{description}

\Subsubsection{Stack objects}{stack-objects-overview}

The stack contains a mixture of \emph{pending arguments} and 
\emph{stack objects}.

Pending arguments are arguments to curried functions which have not
yet been incorporated into an activation frame.  For example, when
evaluating @let { g x y = x + y; f x = g{x} } in f{3,4}@, the
evaluator pushes both arguments onto the stack and enters @f@.  @f@
only requires one argument so it leaves the second argument as a
\emph{pending argument}.  The pending argument remains on the stack
until @f@ calls @g@ which requires two arguments: the argument passed
to it by @f@ and the pending argument which was passed to @f@.

Unboxed pending arguments are always preceeded by a ``tag'' which says
how large the argument is.  This allows the garbage collector to
locate pointers within the stack.

There are three kinds of stack object: return addresses, update frames
and seq frames.  All stack objects look like this

\begin{center}
\begin{tabular}{|l|l|l|l|}\hline
\emph{Header} & \emph{Payload} \\ \hline
\end{tabular}
\end{center}

As with heap objects, the header starts with a pointer to a pair
consisting of an \emph{info table} and some \emph{entry code}.

\begin{description}

\item[Return addresses] are used to cause selection and execution of
case alternatives when a constructor is returned.  Return addresses
generated by the machine code compiler look like this:

\begin{center}
\begin{tabular}{|l|l|l|l|}\hline
@RET_XXX@ & \emph{Free Variables of the case alternatives} \\ \hline
\end{tabular}
\end{center}

The free variables are a mixture of pointers and non-pointers whose
layout is described by a bitmask in the info table.

There are several kinds of @RET_XXX@ return address - see
\secref{activation-records} for the details.

Return addresses generated by the bytecode compiler look like this:
\begin{center}
\begin{tabular}{|l|l|l|l|}\hline
@BCO_RET@ & \emph{BCO} & \emph{Free Variables of the case alternatives} \\ \hline
\end{tabular}
\end{center}

There is just one @BCO_RET@ info pointer.  We avoid needing different
@BCO_RET@s for each stack layout by tagging unboxed free variables as
though they were pending arguments.

\item[Update frames] are used to trigger updates.  When an update
frame is entered, it overwrites the updatee with an indirection to the
result, restarts any threads blocked on the @BLACKHOLE@ and returns to
the stack object underneath the update frame.

\begin{center}
\begin{tabular}{|l|l|l|l|}\hline
@UPDATE_FRAME@ & \emph{Next Update Frame} & \emph{Updatee} \\ \hline
\end{tabular}
\end{center}

\item[Seq frames] are used to implement the polymorphic @seq@
primitive.  They are a special kind of update frame, and are linked on
the update frame list.

\begin{center}
\begin{tabular}{|l|l|l|l|}\hline
@SEQ_FRAME@ & \emph{Next Update Frame} \\ \hline
\end{tabular}
\end{center}

\item[Stop frames] are put on the bottom of each thread's stack, and
act as sentinels for the update frame list (i.e. the last update frame
points to the stop frame).  Returning to a stop frame terminates the
thread.  Stop frames have no payload:

\begin{center}
\begin{tabular}{|l|l|l|l|}\hline
@SEQ_FRAME@ \\ \hline
\end{tabular}
\end{center}

\end{description}

\Subsubsection{Case expressions}{case-expr-overview}

In the STG language, all evaluation is triggered by evaluating a case
expression.  When evaluating a case expression @case e of alts@, the
evaluators pushes a return address onto the stack and evaluate the
expression @e@.  When @e@ eventually reduces to a constructor, the
return address on the stack is entered.  The details of how the
constructor is passed to the return address and how the appropriate
case alternative is selected vary between evaluators.

Case expressions for unboxed data types are essentially the same: the
case expression pushes a return address onto the stack before
evaluating the scrutinee; when a function returns an unboxed value, it
enters the return address on top of the stack.


\Subsubsection{Function applications}{fun-app-overview}

In the STG language, all function calls are tail calls.  The arguments
are pushed onto the stack and the function closure is entered.  If any
arguments are unboxed, they must be tagged as unboxed pending
arguments.  Entering a closure is just a special case of calling a
function with no arguments.


\Subsubsection{Let expressions}{let-expr-overview}

In the STG language, almost all heap allocation is caused by let
expressions.  Filling in the contents of a set of mutually recursive
heap objects is simple enough; the only difficulty is that once the
heap space has been allocated, the thread must not return to the
scheduler until after the objects are filled in.


\Subsubsection{Primitive operations}{primop-overview}

\ToDo{}

Most primops are simple, some aren't.






\Section{Scheduler}{scheduler-overview}

The Scheduler is the heart of the run-time system.  A running program
consists of a single running thread, and a list of runnable and
blocked threads.  A thread is represented by a \emph{Thread Status
Object} (TSO), which contains a few words status information and a
stack.  Except for the running thread, all threads have a closure on
top of their stack; the scheduler restarts a thread by entering an
evaluator which performs some reduction and returns to the scheduler.

\Subsection{The scheduler's main loop}{scheduler-main-loop}

The scheduler consists of a loop which chooses a runnable thread and
invokes one of the evaluators which performs some reduction and
returns.

The scheduler also takes care of system-wide issues such as heap
overflow or communication with other processors (in the parallel
system) and thread-specific problems such as stack overflow.

\Subsection{Creating a thread}{create-thread}

Threads are created:

\begin{itemize}

\item

When the scheduler is first invoked.

\item

When a message is received from another processor (I think). (Parallel
system only.)

\item

When a C program calls some Haskell code.

\item

By @forkIO@, @takeMVar@ and (maybe) other Concurrent Haskell primitives.

\end{itemize}


\Subsection{Restarting a thread}{thread-restart}

When the scheduler decides to run a thread, it has to decide which
evaluator to use.  It does this by looking at the type of the closure
on top of the stack.
\begin{itemize}
\item @BCO@ $\Rightarrow$ bytecode evaluator
\item @FUN@ or @THUNK@ $\Rightarrow$ machine code evaluator
\item @CONSTR@ $\Rightarrow$ machine code evaluator
\item other $\Rightarrow$ either evaluator.
\end{itemize}

The only surprise in the above is that the scheduler must enter the
machine code evaluator if there's a constructor on top of the stack.
This allows the bytecode evaluator to return a constructor to a
machine code return address by pushing the constructor on top of the
stack and returning to the scheduler.  If the return address under the
constructor is @HUGS_RET@, the entry code for @HUGS_RET@ will
rearrange the stack so that the return @BCO@ is on top of the stack
and return to the scheduler which will then call the bytecode
evaluator.  There is little point in trying to shorten this slightly
indirect route since it is will happen very rarely if at all.

\note{As an optimisation, we could store the choice of evaluator in
the TSO status whenever we leave the evaluator.  This is required for
any thread, no matter what state it is in (blocked, stack overflow,
etc).  It isn't clear whether this would accomplish anything.}

\Subsection{Returning from a thread}{thread-return}

The evaluators return to the scheduler when any of the following
conditions arise:

\begin{itemize}
\item A heap check fails, and a garbage collection is required.

\item A stack check fails, and the scheduler must either enlarge the
current thread's stack, or flag an out of memory condition.

\item A thread enters a closure built by the other evaluator.  That
is, when the bytecode interpreter enters a closure compiled by GHC or
when the machine code evaluator enters a BCO.

\item A thread returns to a return continuation built by the other
evaluator.  That is, when the machine code evaluator returns to a
continuation built by Hugs or when the bytecode evaluator returns to a
continuation built by GHC.

\item The evaluator needs to perform a ``safe'' C call
(\secref{c-calls}).

\item The thread becomes blocked.  This happens when a thread requires
the result of a computation currently being performed by another
thread, or it reads a synchronisation variable that is currently empty
(\secref{MVAR}).

\item The thread is preempted (the preemption mechanism is described
in \secref{thread-preemption}).

\item The thread terminates.
\end{itemize}

Except when the thread terminates, the thread always terminates with a
closure on the top of the stack.  The mechanism used to trigger the
world switch and the choice of closure left on top of the stack varies
according to which world is being left and what is being returned.

\Subsubsection{Leaving the bytecode evaluator}{hugs-to-ghc-switch}

\paragraph{Entering a machine code closure}

When it enters a closure, the bytecode evaluator performs a switch
based on the type of closure (@AP@, @PAP@, @Ind@, etc).  On entering a
machine code closure, it returns to the scheduler with the closure on
top of the stack.

\paragraph{Returning a constructor}

When it enters a constructor, the bytecode evaluator tests the return
continuation on top of the stack.  If it is a machine code
continuation, it returns to the scheduler with the constructor on top
of the stack.

\note{This is why the scheduler must enter the machine code evaluator
if it finds a constructor on top of the stack.}

\paragraph{Returning an unboxed value}

\note{Hugs doesn't support unboxed values in source programs but they
are used for a few complex primops.}

When it returns an unboxed value, the bytecode evaluator tests the
return continuation on top of the stack.  If it is a machine code
continuation, it returns to the scheduler with the tagged unboxed
value and a special closure on top of the stack.  When the closure is
entered (by the machine code evaluator), it returns the unboxed value
on top of the stack to the return continuation under it.

The runtime library for GHC provides one of these closures for each unboxed
type.  Hugs cannot generate them itself since the entry code is really
very tricky.

\paragraph{Heap/Stack overflow and preemption}

The bytecode evaluator tests for heap/stack overflow and preemption
when entering a BCO and simply returns with the BCO on top of the
stack.

\Subsubsection{Leaving the machine code evaluator}{ghc-to-hugs-switch}

\paragraph{Entering a BCO}

The entry code for a BCO pushes the BCO onto the stack and returns to
the scheduler.

\paragraph{Returning a constructor}

We avoid the need to test return addresses in the machine code
evaluator by pushing a special return address on top of a pointer to
the bytecode return continuation.  \figref{hugs-return-stack1}
shows the state of the stack just before evaluating the scrutinee.

\begin{figure}[ht]
\begin{center}
\begin{verbatim}
| stack    |
+----------+
| bco      |--> BCO
+----------+
| HUGS_RET |
+----------+
\end{verbatim}
%\input{hugs_return1.pstex_t}
\end{center}
\caption{Stack layout for evaluating a scrutinee}
\label{fig:hugs-return-stack1}
\end{figure}

This return address rearranges the stack so that the bco pointer is
above the constructor on the stack (as shown in
\figref{hugs-boxed-return}) and returns to the scheduler.

\begin{figure}[ht]
\begin{center}
\begin{verbatim}
| stack    |
+----------+
| con      |--> Constructor
+----------+
| bco      |--> BCO
+----------+
\end{verbatim}
%\input{hugs_return2.pstex_t}
\end{center}
\caption{Stack layout for entering a Hugs return address}
\label{fig:hugs-boxed-return}
\end{figure}

\paragraph{Returning an unboxed value}

We avoid the need to test return addresses in the machine code
evaluator by pushing a special return address on top of a pointer to
the bytecode return continuation.  This return address rearranges the
stack so that the bco pointer is above the tagged unboxed value (as
shown in \figref{hugs-entering-unboxed-return}) and returns to the
scheduler.

\begin{figure}[ht]
\begin{center}
\begin{verbatim}
| stack    |
+----------+
| 1#       |
+----------+
| I#       |
+----------+
| bco      |--> BCO
+----------+
\end{verbatim}
%\input{hugs_return2.pstex_t}
\end{center}
\caption{Stack layout for returning an unboxed value}
\label{fig:hugs-entering-unboxed-return}
\end{figure}

\paragraph{Heap/Stack overflow and preemption}

\ToDo{}


\Subsection{Preempting a thread}{thread-preemption}

Strictly speaking, threads cannot be preempted --- the scheduler
merely sets a preemption request flag which the thread must arrange to
test on a regular basis.  When an evaluator finds that the preemption
request flag is set, it pushes an appropriate closure onto the stack
and returns to the scheduler.

In the bytecode interpreter, the flag is tested whenever we enter a
closure.  If the preemption flag is set, it leaves the closure on top
of the stack and returns to the scheduler.

In the machine code evaluator, the flag is only tested when a heap or
stack check fails.  This is less expensive than testing the flag on
entering every closure but runs the risk that a thread will enter an
infinite loop which does not allocate any space.  If the flag is set,
the evaluator returns to the scheduler exactly as if a heap check had
failed.

\Subsection{``Safe'' and ``unsafe'' C calls}{c-calls}

There are two ways of calling C: 

\begin{description}

\item[``Unsafe'' C calls] are used if the programer is certain that
the C function will not do anything dangerous.  Unsafe C calls are
faster but must be hand-checked by the programmer.

Dangerous things include:

\begin{itemize}

\item 

Call a system function such as @getchar@ which might block
indefinitely.  This is dangerous because we don't want the entire
runtime system to block just because one thread blocks.

\item

Call an RTS function which will block on the RTS access semaphore.
This would lead to deadlock.

\item

Call a Haskell function.  This is just a special case of calling an
RTS function.

\end{itemize}

Unsafe C calls are performed by pushing the arguments onto the C stack
and jumping to the C function's entry point.  On exit, the result of
the function is in a register which is returned to the Haskell code as
an unboxed value.

\item[``Safe'' C calls] are used if the programmer suspects that the
thread may do something dangerous.  Safe C calls are relatively slow
but are less problematic.

Safe C calls are performed by pushing the arguments onto the Haskell
stack, pushing a return continuation and returning a \emph{C function
descriptor} to the scheduler.  The scheduler suspends the Haskell thread,
spawns a new operating system thread which pops the arguments off the
Haskell stack onto the C stack, calls the C function, pushes the
function result onto the Haskell stack and informs the scheduler that
the C function has completed and the Haskell thread is now runnable.

\end{description}

The bytecode evaluator will probably treat all C calls as being safe.

\ToDo{It might be good for the programmer to indicate how the program
is unsafe.  For example, if we distinguish between C functions which
might call Haskell functions and those which might block, we could
perform an unsafe call for blocking functions in a single-threaded
system or, perhaps, in a multi-threaded system which only happens to
have a single thread at the moment.}



\Section{The Storage Manager}{sm-overview}

The storage manager is responsible for managing the heap and all
objects stored in it.  It provides special support for lazy evaluation
and for foreign function calls.

\Subsection{SM support for lazy evaluation}{sm-lazy-evaluation}

\begin{itemize}
\item

Indirections are shorted out.

\item

Update frames pointing to unreachable objects are squeezed out.

\ToDo{Part IV suggests this doesn't happen.}

\item

Adjacent update frames (for different closures) are compressed to a
single update frame pointing to a single black hole.

\end{itemize}


\Subsection{SM support for foreign function calls}{sm-foreign-calls}

\begin{itemize}

\item

Stable pointers allow other languages to access Haskell objects.

\item

Weak pointers and foreign objects provide finalisation support for
Haskell references to external objects.

\end{itemize}

\Subsection{Misc}{sm-misc}

\begin{itemize}

\item

If the stack contains a large amount of free space, the storage
manager may shrink the stack.  If it shrinks the stack, it guarantees
never to leave less than @MIN_SIZE_SHRUNKEN_STACK@ empty words on the
stack when it does so.

\item

For efficiency reasons, very large objects (eg large arrays and TSOs)
are not moved if possible.

\end{itemize}


\Section{The Compilers}{compilers-overview}

Need to describe interface files, format of bytecode files, symbols
defined by machine code files.

\Subsection{Interface Files}{interface-files}

Here's an example - but I don't know the grammar - ADR.
\begin{verbatim}
_interface_ Main 1
_exports_
Main main ;
_declarations_
1 main _:_ IOBase.IO PrelBase.();;
\end{verbatim}

\Subsection{Bytecode files}{bytecode-files}

(All that matters here is what the loader sees.)

\Subsection{Machine code files}{asm-files}

(Again, all that matters is what the loader sees.)

\Section{The Loader}{loader-overview}

In a batch mode system, we can statically link all the modules
together.  In an interactive system we need a loader which will
explicitly load and unload individual modules (or, perhaps, blocks of
mutually dependent modules) and resolve references between modules.

While many operating systems provide support for dynamic loading and
will automatically resolve cross-module references for us, we generally
cannot rely on being able to load mutually dependent modules.

A portable solution is to perform some of the linking ourselves.  Each module
should provide three global symbols: 
\begin{itemize}
\item
An initialisation routine.  (Might also be used for finalisation.)
\item
A table of symbols it exports.
Entries in this table consist of the symbol name and the address of the
name's value.
\item
A table of symbols it imports.
Entries in this table consist of the symbol name and a list of references
to that symbol.
\end{itemize}

On loading a group of modules, the loader adds the contents of the
export lists to a symbol table and then fills in all the references in the
import lists.

References in import lists are of two types:
\begin{description}
\item[ References in machine code ]

The most efficient approach is to patch the machine code directly, but
this will be a lot of work, very painful to port and rather fragile.

Alternatively, the loader could store the value of each symbol in the
import table for each module and the compiled code can access all
external objects through the import table.  This requires that the
import table be writable but does not require that the machine code or
info tables be writable.

\item[ References in data structures (SRTs and static data constructors) ]

Either we patch the SRTs and constructors directly or we somehow use
indirections through the symbol table.  Patching the SRTs requires
that we make them writable and prevents us from making effective use
of virtual memories that use copy-on-write policies (this only makes a
difference if we want to run several copies of the same program
simultaneously).  Using an indirection is possible but tricky.

Note: We could avoid patching machine code if all references to
external references went through the SRT --- then we just have one
thing to patch.  But the SRT always contains a pointer to the closure
rather than the fast entry point (say), so we'd take a big performance
hit for doing this.

\end{description}

Using the above scheme, all accesses to ``external'' objects involve a
layer of indirection.  To avoid this overhead, the machine code
compiler might provide a way for the programmer to specify which
modules will be statically linked and which will be dynamically linked
--- the idea being that statically linked code and data will be
accessed directly.


%%%%%%%%%%%%%%%%%%%%%%%%%%%%%%%%%%%%%%%%%%%%%%%%%%%%%%%%%%%%%%%%
\part{Internal details}
%%%%%%%%%%%%%%%%%%%%%%%%%%%%%%%%%%%%%%%%%%%%%%%%%%%%%%%%%%%%%%%%

This part is concerned with the internal details of the components
described in the previous part.

The major components of the system are:
\begin{itemize}
\item The scheduler (\secref{scheduler-internals})
\item The storage manager (\secref{storage-manager-internals})
\item The evaluators
\item The loader
\item The compilers
\end{itemize}

\Section{The Scheduler}{scheduler-internals}

\ToDo{Detailed description of scheduler}

Many heap objects contain fields allowing them to be inserted onto lists
during evaluation or during garbage collection. The lists required by
the evaluator and storage manager are as follows.

\begin{itemize}

\item 4 lists of threads: runnable threads, sleeping threads, threads
waiting for timeout and threads waiting for I/O.

\item The \emph{mutables list} is a list of all objects in the old
generation which might contain pointers into the new generation.  Most
of the objects on this list are indirections (\secref{IND})
or ``mutable.''  (\secref{mutables}.)

\item The \emph{Foreign Object list} is a list of all foreign objects
 which have not yet been deallocated. (\secref{FOREIGN}.)

\item The \emph{Spark pool} is a doubly(?) linked list of Spark objects
maintained by the parallel system.  (\secref{SPARK}.)

\item The \emph{Blocked Fetch list} (or
lists?). (\secref{BLOCKED_FETCH}.)

\item For each thread, there is a list of all update frames on the
stack.  (\secref{data-updates}.)

\item The Stable Pointer Table is a table of pointers to objects which
are known to the outside world and must be retained by the garbage
collector even if they are not accessible from within the heap.

\end{itemize}

\ToDo{The links for these fields are usually inserted immediately
after the fixed header except ...}



\Section{The Storage Manager}{storage-manager-internals}

\subsection{Misc Text looking for a home}

A \emph{value} may be:
\begin{itemize}
\item \emph{Boxed}, i.e.~represented indirectly by a pointer to a heap object (e.g.~foreign objects, arrays); or
\item \emph{Unboxed}, i.e.~represented directly by a bit-pattern in one or more registers (e.g.~@Int#@ and @Float#@).
\end{itemize}
All \emph{pointed} values are \emph{boxed}.  


\Subsection{Heap Objects}{heap-objects}
\label{sec:fixed-header}

\begin{figure}
\begin{center}
\makebox[3.597in][l]{
  \vbox to 2.375in{
    \vfill
    \special{psfile=closure.ps}
  }
  \vspace{-\baselineskip}
}

\end{center}
\ToDo{Fix this picture}
\caption{A closure}
\label{fig:closure}
\end{figure}

Every \emph{heap object} is a contiguous block of memory, consisting
of a fixed-format \emph{header} followed by zero or more \emph{data
words}.

The header consists of the following fields:
\begin{itemize}
\item A one-word \emph{info pointer}, which points to
the object's static \emph{info table}.
\item Zero or more \emph{admin words} that support
\begin{itemize}
\item Profiling (notably a \emph{cost centre} word).
  \note{We could possibly omit the cost centre word from some 
  administrative objects.}
\item Parallelism (e.g. GranSim keeps the object's global address here,
though GUM keeps a separate hash table).
\item Statistics (e.g. a word to track how many times a thunk is entered.).

We add a Ticky word to the fixed-header part of closures.  This is
used to indicate if a closure has been updated but not yet entered. It
is set when the closure is updated and cleared when subsequently
entered.  \footnote{% NB: It is \emph{not} an ``entry count'', it is
an ``entries-after-update count.''  The commoning up of @CONST@,
@CHARLIKE@ and @INTLIKE@ closures is turned off(?) if this is
required. This has only been done for 2s collection.  }

\end{itemize}
\end{itemize}

Most of the RTS is completely insensitive to the number of admin
words.  The total size of the fixed header is given by
@sizeof(StgHeader)@.

\Subsection{Info Tables}{info-tables}

An \emph{info table} is a contiguous block of memory, laid out as follows:

\begin{center}
\begin{tabular}{|r|l|}
   \hline Parallelism Info 	& variable
\\ \hline Profile Info 		& variable
\\ \hline Debug Info		& variable
\\ \hline Static reference table  & pointer word (optional)
\\ \hline Storage manager layout info & pointer word
\\ \hline Closure flags		& 8 bits
\\ \hline Closure type 		& 8 bits
\\ \hline Constructor Tag / SRT length    	& 16 bits
\\ \hline entry code
\\       \vdots
\end{tabular}
\end{center}

On a 64-bit machine the tag, type and flags fields will all be doubled
in size, so the info table is a multiple of 64 bits.

An info table has the following contents (working backwards in memory
addresses):

\begin{itemize}

\item The \emph{entry code} for the closure.  This code appears
literally as the (large) last entry in the info table, immediately
preceded by the rest of the info table.  An \emph{info pointer} always
points to the first byte of the entry code.

\item A 16-bit constructor tag / SRT length.  For a constructor info
table this field contains the tag of the constructor, in the range
$0..n-1$ where $n$ is the number of constructors in the datatype.
Otherwise, it contains the number of entries in this closure's Static
Reference Table (\secref{srt}).

\item An 8-bit {\em closure type field}, which identifies what kind of
closure the object is.  The various types of closure are described in
\secref{closures}.

\item an 8-bit flags field, which holds various flags pertaining to
the closure type.

\item A single pointer or word --- the {\em storage manager info
field}, contains auxiliary information describing the closure's
precise layout, for the benefit of the garbage collector and the code
that stuffs graph into packets for transmission over the network.
There are three kinds of layout information:

\begin{itemize}
\item Standard layout information is for closures which place pointers
before non-pointers in instances of the closure (this applies to most
heap-based and static closures, but not activation records).  The
layout information for standard closures is

	\begin{itemize}
	\item Number of pointer fields (16 bits).
	\item Number of non-pointer fields (16 bits).
	\end{itemize}

\item Activation records don't have pointers before non-pointers,
since stack-stubbing requires that the record has holes in it.  The
layout is therefore represented by a bitmap in which each '1' bit
represents a non-pointer word.  This kind of layout info is used for
@RET_SMALL@ and @RET_VEC_SMALL@ closures.

\item If an activation record is longer than 32 words, then the layout
field contains a pointer to a bitmap record, consisting of a length
field followed by two or more bitmap words.  This layout information
is used for @RET_BIG@ and @RET_VEC_BIG@ closures.

\item Selector Thunks (\secref{THUNK_SELECTOR}) use the closure
layout field to hold the selector index, since the layout is always
known (the closure contains a single pointer field).
\end{itemize}

\item A one-word {\em Static Reference Table} field.  This field
points to the static reference table for the closure (\secref{srt}),
and is only present for the following closure types:

	\begin{itemize}
	\item @FUN_*@
	\item @THUNK_*@
	\item @RET_*@
	\end{itemize}

\ToDo{Expand the following explanation.}

An SRT is basically a vector of pointers to static closures.  A
top-level function or thunk will have an SRT (which might be empty),
which points to all the static closures referenced by that function or
thunk.  Every non-top-level thunk or function also has an SRT, but
it'll be a sub-sequence of the top-level SRT, so we just store a
pointer and a length in the info table - the pointer points into the
middle of the larger SRT.

At GC time, the garbage collector traverses the transitive closure of
all the SRTs reachable from the roots, and thereby discovers which
CAFs are live.
  
\item \emph{Profiling info\/}

\ToDo{The profiling info is completely bogus.  I've not deleted it
from the document but I've commented it all out.}

% change to \iftrue to uncomment this section
\iffalse

Closure category records are attached to the info table of the
closure. They are declared with the info table. We put pointers to
these ClCat things in info tables.  We need these ClCat things because
they are mutable, whereas info tables are immutable.  Hashing will map
similar categories to the same hash value allowing statistics to be
grouped by closure category.

Cost Centres and Closure Categories are hashed to provide indexes
against which arbitrary information can be stored. These indexes are
memoised in the appropriate cost centre or category record and
subsequent hashes avoided by the index routine (it simply returns the
memoised index).

There are different features which can be hashed allowing information
to be stored for different groupings. Cost centres have the cost
centre recorded (using the pointer), module and group. Closure
categories have the closure description and the type
description. Records with the same feature will be hashed to the same
index value.

The initialisation routines, @init_index_<feature>@, allocate a hash
table in which the cost centre / category records are stored. The
lower bound for the table size is taken from @max_<feature>_no@. They
return the actual table size used (the next power of 2). Unused
locations in the hash table are indicated by a 0 entry. Successive
@init_index_<feature>@ calls just return the actual table size.

Calls to @index_<feature>@ will insert the cost centre / category
record in the @<feature>@ hash table, if not already inserted. The hash
index is memoised in the record and returned. 

CURRENTLY ONLY ONE MEMOISATION SLOT IS AVILABLE IN EACH RECORD SO
HASHING CAN ONLY BE DONE ON ONE FEATURE FOR EACH RECORD. This can be
easily relaxed at the expense of extra memoisation space or continued
rehashing.

The initialisation routines must be called before initialisation of
the stacks and heap as they require to allocate storage. It is also
expected that the caller may want to allocate additional storage in
which to store profiling information based on the return table size
value(s).

\begin{center}
\begin{tabular}{|l|}
   \hline Hash Index
\\ \hline Selected
\\ \hline Kind
\\ \hline Description String
\\ \hline Type String
\\ \hline
\end{tabular}
\end{center}

\begin{description}
\item[Hash Index] Memoised copy
\item[Selected] 
  Is this category selected (-1 == not memoised, selected? 0 or 1)
\item[Kind]
One of the following values (defined in CostCentre.lh):

\begin{description}
\item[@CON_K@]
A constructor.
\item[@FN_K@]
A literal function.
\item[@PAP_K@]
A partial application.
\item[@THK_K@]
A thunk, or suspension.
\item[@BH_K@]
A black hole.
\item[@ARR_K@]
An array.
\item[@ForeignObj_K@]
A Foreign object (non-Haskell heap resident).
\item[@SPT_K@]
The Stable Pointer table.  (There should only be one of these but it
represents a form of weak space leak since it can't shrink to meet
non-demand so it may be worth watching separately? ADR)
\item[@INTERNAL_KIND@]
Something internal to the runtime system.
\end{description}


\item[Description] Source derived string detailing closure description.
\item[Type] Source derived string detailing closure type.
\end{description}

\fi % end of commented out stuff

\item \emph{Parallelism info\/}
\ToDo{}

\item \emph{Debugging info\/}
\ToDo{}

\end{itemize}


%-----------------------------------------------------------------------------
\Subsection{Kinds of Heap Object}{closures}

Heap objects can be classified in several ways, but one useful one is
this:
\begin{itemize}
\item 
\emph{Static closures} occupy fixed, statically-allocated memory
locations, with globally known addresses.

\item 
\emph{Dynamic closures} are individually allocated in the heap.

\item 
\emph{Stack closures} are closures allocated within a thread's stack
(which is itself a heap object).  Unlike other closures, there are
never any pointers to stack closures.  Stack closures are discussed in
\secref{TSO}.

\end{itemize}
A second useful classification is this:
\begin{itemize}

\item \emph{Executive objects}, such as thunks and data constructors,
participate directly in a program's execution.  They can be subdivided
into three kinds of objects according to their type: \begin{itemize}

\item \emph{Pointed objects}, represent values of a \emph{pointed}
type (<.pointed types launchbury.>) --i.e.~a type that includes
$\bottom$ such as @Int@ or @Int# -> Int#@.

\item \emph{Unpointed objects}, represent values of a \emph{unpointed}
type --i.e.~a type that does not include $\bottom$ such as @Int#@ or
@Array#@.

\item \emph{Activation frames}, represent ``continuations''.  They are
always stored on the stack and are never pointed to by heap objects or
passed as arguments.  \note{It's not clear if this will still be true
once we support speculative evaluation.}

\end{itemize}

\item \emph{Administrative objects}, such as stack objects and thread
state objects, do not represent values in the original program.
\end{itemize}

Only pointed objects can be entered.  If an unpointed object is
entered the program will usually terminate with a fatal error.

This section enumerates all the kinds of heap objects in the system.
Each is identified by a distinct closure type field in its info table.

\begin{tabular}{|l|l|l|l|l|l|l|l|l|l|l|}
\hline

closure type          & Section \\
		      
\hline                          
\emph{Pointed} \\      
\hline 		      
		      
@CONSTR@              & \ref{sec:CONSTR}    \\
@CONSTR_p_n@	      & \ref{sec:CONSTR}    \\
@CONSTR_STATIC@       & \ref{sec:CONSTR}    \\
@CONSTR_NOCAF_STATIC@ & \ref{sec:CONSTR}    \\
		      
@FUN@                 & \ref{sec:FUN}       \\
@FUN_p_n@             & \ref{sec:FUN}       \\
@FUN_STATIC@          & \ref{sec:FUN}       \\
		      
@THUNK@               & \ref{sec:THUNK}     \\
@THUNK_p_n@           & \ref{sec:THUNK}     \\
@THUNK_STATIC@        & \ref{sec:THUNK}     \\
@THUNK_SELECTOR@      & \ref{sec:THUNK_SELECTOR} \\
		      
@BCO@		      & \ref{sec:BCO}       \\
		      
@AP_UPD@      	      & \ref{sec:AP_UPD}    \\
@PAP@                 & \ref{sec:PAP}       \\
		      
@IND@                 & \ref{sec:IND}       \\
@IND_OLDGEN@          & \ref{sec:IND}       \\
@IND_PERM@            & \ref{sec:IND}       \\
@IND_OLDGEN_PERM@     & \ref{sec:IND}       \\
@IND_STATIC@          & \ref{sec:IND}       \\
		      
@CAF_UNENTERED@	      & \ref{sec:CAF}       \\
@CAF_ENTERED@	      & \ref{sec:CAF}       \\
@CAF_BLACKHOLE@	      & \ref{sec:CAF}       \\

\hline		      
\emph{Unpointed} \\    
\hline		      
		      		      
@BLACKHOLE@           & \ref{sec:BLACKHOLE} \\
@BLACKHOLE_BQ@        & \ref{sec:BLACKHOLE_BQ} \\

@MVAR@ 		      & \ref{sec:MVAR}      \\

@ARR_WORDS@           & \ref{sec:ARR_WORDS} \\

@MUTARR_PTRS@         & \ref{sec:MUT_ARR_PTRS} \\
@MUTARR_PTRS_FROZEN@  & \ref{sec:MUT_ARR_PTRS_FROZEN} \\

@MUT_VAR@              & \ref{sec:MUT_VAR}    \\

@WEAK@		      & \ref{sec:WEAK}   \\
@FOREIGN@             & \ref{sec:FOREIGN}   \\
@STABLE_NAME@	      & \ref{sec:STABLE_NAME}   \\
\hline
\end{tabular}

Activation frames do not live (directly) on the heap --- but they have
a similar organisation.

\begin{tabular}{|l|l|}\hline
closure type		& Section			\\ \hline
@RET_SMALL@ 		& \ref{sec:activation-records}	\\
@RET_VEC_SMALL@ 	& \ref{sec:activation-records}	\\
@RET_BIG@		& \ref{sec:activation-records}	\\
@RET_VEC_BIG@		& \ref{sec:activation-records}	\\
@UPDATE_FRAME@ 		& \ref{sec:activation-records}	\\
@CATCH_FRAME@ 		& \ref{sec:activation-records}	\\
@SEQ_FRAME@ 		& \ref{sec:activation-records}	\\
@STOP_FRAME@ 		& \ref{sec:activation-records}	\\
\hline
\end{tabular}

There are also a number of administrative objects.  It is an error to
enter one of these objects.

\begin{tabular}{|l|l|}\hline
closure type		& Section 			\\ \hline
@TSO@                   & \ref{sec:TSO} 		\\
@SPARK_OBJECT@          & \ref{sec:SPARK}		\\
@BLOCKED_FETCH@       	& \ref{sec:BLOCKED_FETCH} 	\\
@FETCHME@               & \ref{sec:FETCHME}   \\
\hline
\end{tabular}

\Subsection{Predicates}{closure-predicates}

The runtime system sometimes needs to be able to distinguish objects
according to their properties: is the object updateable? is it in weak
head normal form? etc.  These questions can be answered by examining
the closure type field of the object's info table.  

We define the following predicates to detect families of related
info types.  They are mutually exclusive and exhaustive.

\begin{itemize}
\item @isCONSTR@ is true for @CONSTR@s.
\item @isFUN@ is true for @FUN@s.
\item @isTHUNK@ is true for @THUNK@s.
\item @isBCO@ is true for @BCO@s.
\item @isAP@ is true for @AP@s.
\item @isPAP@ is true for @PAP@s.
\item @isINDIRECTION@ is true for indirection objects. 
\item @isBH@ is true for black holes.
\item @isFOREIGN_OBJECT@ is true for foreign objects.
\item @isARRAY@ is true for array objects.
\item @isMVAR@ is true for @MVAR@s.
\item @isIVAR@ is true for @IVAR@s.
\item @isFETCHME@ is true for @FETCHME@s.
\item @isSLOP@ is true for slop objects.
\item @isRET_ADDR@ is true for return addresses.
\item @isUPD_ADDR@ is true for update frames.
\item @isTSO@ is true for @TSO@s.
\item @isSTABLE_PTR_TABLE@ is true for the stable pointer table.
\item @isSPARK_OBJECT@ is true for spark objects.
\item @isBLOCKED_FETCH@ is true for blocked fetch objects.
\item @isINVALID_INFOTYPE@ is true for all other info types.

\end{itemize}

The following predicates detect other interesting properties:

\begin{itemize}

\item @isPOINTED@ is true if an object has a pointed type.

If an object is pointed, the following predicates may be true
(otherwise they are false).  @isWHNF@ and @isUPDATEABLE@ are
mutually exclusive.

\begin{itemize} 
\item @isWHNF@ is true if the object is in Weak Head Normal Form.  
Note that unpointed objects are (arbitrarily) not considered to be in WHNF.

@isWHNF@ is true for @PAP@s, @CONSTR@s, @FUN@s and all @BCO@s.

\ToDo{Need to distinguish between whnf BCOs and non-whnf BCOs in their
closure type}

\item @isUPDATEABLE@ is true if the object may be overwritten with an
 indirection object.

@isUPDATEABLE@ is true for @THUNK@s, @AP@s and @BH@s.

\end{itemize}

It is possible for a pointed object to be neither updatable nor in
WHNF.  For example, indirections.

\item @isUNPOINTED@ is true if an object has an unpointed type.
All such objects are boxed since only boxed objects have info pointers.

It is true for @ARR_WORDS@, @ARR_PTRS@, @MUTVAR@, @MUTARR_PTRS@,
@MUTARR_PTRS_FROZEN@, @FOREIGN@ objects, @MVAR@s and @IVAR@s.

\item @isACTIVATION_FRAME@ is true for activation frames of all sorts.

It is true for return addresses and update frames.
\begin{itemize}
\item @isVECTORED_RETADDR@ is true for vectored return addresses.
\item @isDIRECT_RETADDR@ is true for direct return addresses.
\end{itemize}

\item @isADMINISTRATIVE@ is true for administrative objects:
@TSO@s, the stable pointer table, spark objects and blocked fetches.

\item @hasSRT@ is true if the info table for the object contains an
SRT pointer.  

@hasSRT@ is true for @THUNK@s, @FUN@s, and @RET@s.

\end{itemize}

\begin{itemize}

\item @isMUTABLE@ is true for objects with mutable pointer fields:
  @MUT_ARR@s, @MUTVAR@s, @MVAR@s and @IVAR@s.

\item @isSparkable@ is true if the object can (and should) be sparked.
It is true of updateable objects which are not in WHNF with the
exception of @THUNK_SELECTOR@s and black holes.

\end{itemize}

As a minor optimisation, we might use the top bits of the @INFO_TYPE@
field to ``cache'' the answers to some of these predicates.

An indirection either points to HNF (post update); or is result of
overwriting a FetchMe, in which case the thing fetched is either under
evaluation (BLACKHOLE), or by now an HNF.  Thus, indirections get
NoSpark flag.

\subsection{Closures (aka Pointed Objects)}

An object can be entered iff it is a closure.

\Subsubsection{Function closures}{FUN}

Function closures represent lambda abstractions.  For example,
consider the top-level declaration:
\begin{verbatim}
  f = \x -> let g = \y -> x+y
	    in g x
\end{verbatim}
Both @f@ and @g@ are represented by function closures.  The closure
for @f@ is \emph{static} while that for @g@ is \emph{dynamic}.

The layout of a function closure is as follows:
\begin{center}
\begin{tabular}{|l|l|l|l|}\hline
\emph{Fixed header}  & \emph{Pointers} & \emph{Non-pointers} \\ \hline
\end{tabular}
\end{center}

The data words (pointers and non-pointers) are the free variables of
the function closure.  The number of pointers and number of
non-pointers are stored in @info->layout.ptrs@ and
@info->layout.nptrs@ respecively.

There are several different sorts of function closure, distinguished
by their closure type field:

\begin{itemize}

\item @FUN@: a vanilla, dynamically allocated on the heap.

\item $@FUN_@p@_@np$: to speed up garbage collection a number of
specialised forms of @FUN@ are provided, for particular $(p,np)$
pairs, where $p$ is the number of pointers and $np$ the number of
non-pointers.

\item @FUN_STATIC@.  Top-level, static, function closures (such as @f@
above) have a different layout than dynamic ones:

\begin{center}
\begin{tabular}{|l|l|l|}\hline
\emph{Fixed header}  & \emph{Static object link} \\ \hline
\end{tabular}
\end{center}

Static function closures have no free variables.  (However they may
refer to other static closures; these references are recorded in the
function closure's SRT.)  They have one field that is not present in
dynamic closures, the \emph{static object link} field.  This is used
by the garbage collector in the same way that to-space is, to gather
closures that have been determined to be live but that have not yet
been scavenged.

\note{Static function closures that have no static references, and
hence a null SRT pointer, don't need the static object link field.  We
don't take advantage of this at the moment, but we could.  See
@CONSTR\_NOCAF\_STATIC@.}  
\end{itemize}

Each lambda abstraction, $f$, in the STG program has its own private
info table.  The following labels are relevant:

\begin{itemize}

\item $f$@_info@  is $f$'s info table.

\item $f$@_entry@ is $f$'s slow entry point (i.e. the entry code of
its info table; so it will label the same byte as $f$@_info@).

\item $f@_fast_@k$ is $f$'s fast entry point.  $k$ is the number of
arguments $f$ takes; encoding this number in the fast-entry label
occasionally catches some nasty code-generation errors.

\end{itemize}

\Subsubsection{Data constructors}{CONSTR}

Data-constructor closures represent values constructed with algebraic
data type constructors.  The general layout of data constructors is
the same as that for function closures.  That is

\begin{center}
\begin{tabular}{|l|l|l|l|}\hline
\emph{Fixed header}  & \emph{Pointers} & \emph{Non-pointers} \\ \hline
\end{tabular}
\end{center}

There are several different sorts of constructor:

\begin{itemize}

\item @CONSTR@: a vanilla, dynamically allocated constructor.

\item @CONSTR_@$p$@_@$np$: just like $@FUN_@p@_@np$.

\item @CONSTR_INTLIKE@.  A dynamically-allocated heap object that
looks just like an @Int@.  The garbage collector checks to see if it
can common it up with one of a fixed set of static int-like closures,
thus getting it out of the dynamic heap altogether.

\item @CONSTR_CHARLIKE@:  same deal, but for @Char@.

\item @CONSTR_STATIC@ is similar to @FUN_STATIC@, with the
complication that the layout of the constructor must mimic that of a
dynamic constructor, because a static constructor might be returned to
some code that unpacks it.  So its layout is like this:

\begin{center}
\begin{tabular}{|l|l|l|l|l|}\hline
\emph{Fixed header}  & \emph{Pointers} & \emph{Non-pointers} & \emph{Static object link}\\ \hline
\end{tabular}
\end{center}

The static object link, at the end of the closure, serves the same purpose
as that for @FUN_STATIC@.  The pointers in the static constructor can point
only to other static closures.

The static object link occurs last in the closure so that static
constructors can store their data fields in exactly the same place as
dynamic constructors.

\item @CONSTR_NOCAF_STATIC@.  A statically allocated data constructor
that guarantees not to point (directly or indirectly) to any CAF
(\secref{CAF}).  This means it does not need a static object
link field.  Since we expect that there might be quite a lot of static
constructors this optimisation makes sense.  Furthermore, the @NOCAF@
tag allows the compiler to indicate that no CAFs can be reached
anywhere \emph{even indirectly}.

\end{itemize}

For each data constructor $Con$, two info tables are generated:

\begin{itemize}
\item $Con$@_con_info@ labels $Con$'s dynamic info table, 
shared by all dynamic instances of the constructor.
\item $Con$@_static@ labels $Con$'s static info table, 
shared by all static instances of the constructor.
\end{itemize}

Each constructor also has a \emph{constructor function}, which is a
curried function which builds an instance of the constructor.  The
constructor function has an info table labelled as @$Con$_info@, and
entry code pointed to by @$Con$_entry@.

Nullary constructors are represented by a single static info table,
which everyone points to.  Thus for a nullary constructor we can omit
the dynamic info table and the constructor function.

\subsubsection{Thunks}
\label{sec:THUNK}
\label{sec:THUNK_SELECTOR}

A thunk represents an expression that is not obviously in head normal 
form.  For example, consider the following top-level definitions:
\begin{verbatim}
  range = between 1 10
  f = \x -> let ys = take x range
	    in sum ys
\end{verbatim}
Here the right-hand sides of @range@ and @ys@ are both thunks; the former
is static while the latter is dynamic.

The layout of a thunk is the same as that for a function closure.
However, thunks must have a payload of at least @MIN_UPD_SIZE@
words to allow it to be overwritten with a black hole and an
indirection.  The compiler may have to add extra non-pointer fields to
satisfy this constraint.

\begin{center}
\begin{tabular}{|l|l|l|l|l|}\hline
\emph{Fixed header}  & \emph{Pointers} & \emph{Non-pointers} \\ \hline
\end{tabular}
\end{center}

The layout word in the info table contains the same information as for
function closures; that is, number of pointers and number of
non-pointers.

A thunk differs from a function closure in that it can be updated.

There are several forms of thunk:

\begin{itemize}

\item @THUNK@ and $@THUNK_@p@_@np$: vanilla, dynamically allocated
thunks.  Dynamic thunks are overwritten with normal indirections
(@IND@), or old generation indirections (@IND_OLDGEN@): see
\secref{IND}.

\item @THUNK_STATIC@.  A static thunk is also known as a
\emph{constant applicative form}, or \emph{CAF}.  Static thunks are
overwritten with static indirections.

\begin{center}
\begin{tabular}{|l|l|}\hline
\emph{Fixed header}  & \emph{Static object link}\\ \hline
\end{tabular}
\end{center}

\item @THUNK_SELECTOR@ is a (dynamically allocated) thunk whose entry
code performs a simple selection operation from a data constructor
drawn from a single-constructor type.  For example, the thunk
\begin{verbatim}
	x = case y of (a,b) -> a
\end{verbatim}
is a selector thunk.  A selector thunk is laid out like this:

\begin{center}
\begin{tabular}{|l|l|l|l|}\hline
\emph{Fixed header}  & \emph{Selectee pointer} \\ \hline
\end{tabular}
\end{center}

The layout word contains the byte offset of the desired word in the
selectee.  Note that this is different from all other thunks.

The garbage collector ``peeks'' at the selectee's tag (in its info
table).  If it is evaluated, then it goes ahead and does the
selection, and then behaves just as if the selector thunk was an
indirection to the selected field.  If it is not evaluated, it treats
the selector thunk like any other thunk of that shape.
[Implementation notes.  Copying: only the evacuate routine needs to be
special.  Compacting: only the PRStart (marking) routine needs to be
special.]

There is a fixed set of pre-compiled selector thunks built into the
RTS, representing offsets from 0 to @MAX_SPEC_SELECTOR_THUNK@.  The
info tables are labelled @__sel_$n$_upd_info@ where $n$ is the offset.
Non-updating versions are also built in, with info tables labelled
@__sel_$n$_noupd_info@.

\end{itemize}

The only label associated with a thunk is its info table:

\begin{description}
\item[$f$@\_info@] is $f$'s info table.
\end{description}


\Subsubsection{Byte-code objects}{BCO}

A Byte-Code Object (BCO) is a container for a chunk of byte-code,
which can be executed by Hugs.  The byte-code represents a
supercombinator in the program: when Hugs compiles a module, it
performs lambda lifting and each resulting supercombinator becomes a
byte-code object in the heap.

BCOs are not updateable; the bytecode compiler represents updatable
thunks using a combination of @AP@s and @BCO@s.

The semantics of BCOs are described in \secref{hugs-heap-objects}.  A
BCO has the following structure:

\begin{center}
\begin{tabular}{|l|l|l|l|l|l|}
\hline 
\emph{Fixed Header} & \emph{Layout} & \emph{Offset} & \emph{Size} &
\emph{Literals} & \emph{Byte code} \\
\hline
\end{tabular}
\end{center}

\noindent where:
\begin{itemize}
\item The entry code is a static code fragment/info table that returns
to the scheduler to invoke Hugs (\secref{ghc-to-hugs-switch}).
\item \emph{Layout} contains the number of pointer literals in the
\emph{Literals} field.
\item \emph{Offset} is the offset to the byte code from the start of
the object.
\item \emph{Size} is the number of words of byte code in the object.
\item \emph{Literals} contains any pointer and non-pointer literals used in
the byte-codes (including jump addresses), pointers first.
\item \emph{Byte code} contains \emph{Size} words of non-pointer byte
code.
\end{itemize}


\Subsubsection{Partial applications}{PAP}

A partial application (PAP) represents a function applied to too few
arguments.  It is only built as a result of updating after an
argument-satisfaction check failure.  A PAP has the following shape:

\begin{center}
\begin{tabular}{|l|l|l|l|}\hline
\emph{Fixed header}  & \emph{No of words of stack} & \emph{Function closure} & \emph{Stack chunk ...} \\ \hline
\end{tabular}
\end{center}

The ``Stack chunk'' is a copy of the chunk of stack above the update
frame; ``No of words of stack'' tells how many words it consists of.
The function closure is (a pointer to) the closure for the function
whose argument-satisfaction check failed.

In the normal case where a PAP is built as a result of an argument
satisfaction check failure, the stack chunk will just contain
``pending arguments'', ie. pointers and tagged non-pointers.  It may
in fact also contain activation records, but not update frames, seq
frames, or catch frames.  The reason is the garbage collector uses the
same code to scavenge a stack as it does to scavenge the payload of a
PAP, but an update frame contains a link to the next update frame in
the chain and this link would need to be relocated during garbage
collection.  Revertible black holes and asynchronous exceptions use
the more general form of PAPs (see Section \ref{revertible-bh}).

There is just one standard form of PAP. There is just one info table
too, called @PAP_info@.  Its entry code simply copies the arg stack
chunk back on top of the stack and enters the function closure.  (It
has to do a stack overflow test first.)

There is just one way to build a PAP: by calling @stg_update_PAP@ with
the function closure in register @R1@ and the pending arguments on the
stack.  The @stg_update_PAP@ function will build the PAP, perform the
update, and return to the next activation record on the stack.  If
there are \emph{no} pending arguments on the stack, then no PAP need
be built: in this case @stg_update_PAP@ just overwrites the updatee
with an indirection to the function closure.

PAPs are also used to implement Hugs functions (where the arguments
are free variables).  PAPs generated by Hugs can be static so we need
both @PAP@ and @PAP_STATIC@.

\Subsubsection{\texttt{AP\_UPD} objects}{AP_UPD}

@AP_UPD@ objects are used to represent thunks built by Hugs, and to
save the currently-active computations when performing @raiseAsync()@.
The only
distinction between an @AP_UPD@ and a @PAP@ is that an @AP_UPD@ is
updateable.

\begin{center}
\begin{tabular}{|l|l|l|l|}
\hline
\emph{Fixed Header} & \emph{No of stack words} & \emph{Function closure} & \emph{Stack chunk} \\
\hline
\end{tabular}
\end{center}

The entry code pushes an update frame, copies the arg stack chunk on
top of the stack, and enters the function closure.  (It has to do a
stack overflow test first.)

The ``stack chunk'' is a block of stack not containing update frames,
seq frames or catch frames (just like a PAP).  In the case of Hugs,
the stack chunk will contain the free variables of the thunk, and the
function closure is (a pointer to) the closure for the thunk.  The
argument stack may be empty if the thunk has no free variables.

\note{Since @AP\_UPD@s are updateable, the @MIN\_UPD\_SIZE@ constraint applies here too.}

\Subsubsection{Indirections}{IND}

Indirection closures just point to other closures. They are introduced
when a thunk is updated to point to its value.  The entry code for all
indirections simply enters the closure it points to.

There are several forms of indirection:

\begin{description}
\item[@IND@] is the vanilla, dynamically-allocated indirection.
It is removed by the garbage collector. It has the following
shape:
\begin{center}
\begin{tabular}{|l|l|l|}\hline
\emph{Fixed header} & \emph{Target closure} \\ \hline
\end{tabular}
\end{center}

An @IND@ only exists in the youngest generation.  In older
generations, we have @IND_OLDGEN@s.  The update code
(@Upd_frame_$n$_entry@) checks whether the updatee is in the youngest
generation before deciding which kind of indirection to use.

\item[@IND\_OLDGEN@] is the vanilla, dynamically-allocated indirection.
It is removed by the garbage collector. It has the following
shape:
\begin{center}
\begin{tabular}{|l|l|l|}\hline
\emph{Fixed header} & \emph{Target closure} & \emph{Mutable link field} \\ \hline
\end{tabular}
\end{center}
It contains a \emph{mutable link field} that is used to string together
mutable objects in each old generation.

\item[@IND\_PERM@]
For lexical profiling, it is necessary to maintain cost centre
information in an indirection, so ``permanent indirections'' are
retained forever.  Otherwise they are just like vanilla indirections.
\note{If a permanent indirection points to another permanent
indirection or a @CONST@ closure, it is possible to elide the indirection
since it will have no effect on the profiler.}

\note{Do we still need @IND@ in the profiling build, or do we just
need @IND@ but its behaviour changes when profiling is on?}

\item[@IND\_OLDGEN\_PERM@]
Just like an @IND_OLDGEN@, but sticks around like an @IND_PERM@.

\item[@IND\_STATIC@] is used for overwriting CAFs when they have been
evaluated.  Static indirections are not removed by the garbage
collector; and are statically allocated outside the heap (and should
stay there).  Their static object link field is used just as for
@FUN_STATIC@ closures.

\begin{center}
\begin{tabular}{|l|l|l|}
\hline
\emph{Fixed header} & \emph{Target closure} & \emph{Static link field} \\
\hline
\end{tabular}
\end{center}

\end{description}

\subsubsection{Black holes and blocking queues}
\label{sec:BLACKHOLE}
\label{sec:BLACKHOLE_BQ}

Black hole closures are used to overwrite closures currently being
evaluated. They inform the garbage collector that there are no live
roots in the closure, thus removing a potential space leak.  

Black holes also become synchronization points in the concurrent
world.  When a thread attempts to enter a blackhole, it must wait for
the result of the computation, which is presumably in progress in
another thread.

\note{In a single-threaded system, entering a black hole indicates an
infinite loop.  In a concurrent system, entering a black hole
indicates an infinite loop only if the hole is being entered by the
same thread that originally entered the closure.  It could also bring
about a deadlock situation where several threads are waiting
circularly on computations in progress.}

There are two types of black hole:

\begin{description}

\item[@BLACKHOLE@]
A straightforward blackhole just consists of an info pointer and some
padding to allow updating with an @IND_OLDGEN@ if necessary.  This
type of blackhole has no waiting threads.

\begin{center}
\begin{tabular}{|l|l|l|}
\hline 
\emph{Fixed header} & \emph{Padding} & \emph{Padding} \\
\hline
\end{tabular}
\end{center}

If we're doing \emph{eager blackholing} then a thunk's info pointer is
overwritten with @BLACKHOLE_info@ at the time of entry; hence the need
for blackholes to be small, otherwise we'd be overwriting part of the
thunk itself.

\item[@BLACKHOLE\_BQ@] 
When a thread enters a @BLACKHOLE@, it is turned into a @BLACKHOLE_BQ@
(blocking queue), which contains a linked list of blocked threads in
addition to the info pointer.

\begin{center}
\begin{tabular}{|l|l|l|}
\hline 
\emph{Fixed header} & \emph{Blocked thread link} & \emph{Mutable link field} \\
\hline
\end{tabular}
\end{center}

The \emph{Blocked thread link} points to the TSO of the first thread
waiting for the value of this thunk.  All subsequent TSOs in the list
are linked together using their @tso->link@ field, ending in
@END_TSO_QUEUE_closure@.

Because new threads can be added to the \emph{Blocked thread link}, a
blocking queue is \emph{mutable}, so we need a mutable link field in
order to chain it on to a mutable list for the generational garbage
collector.

\end{description}

\Subsubsection{FetchMes}{FETCHME} 

In the parallel systems, FetchMes are used to represent pointers into
the global heap.  When evaluated, the value they point to is read from
the global heap.

\ToDo{Describe layout}

Because there may be offsets into these arrays, a primitive array
cannot be handled as a FetchMe in the parallel system, but must be
shipped in its entirety if its parent closure is shipped.



\Subsection{Unpointed Objects}{unpointed-objects}

A variable of unpointed type is always bound to a \emph{value}, never
to a \emph{thunk}.  For this reason, unpointed objects cannot be
entered.

\subsubsection{Immutable objects}
\label{sec:ARR_WORDS}

\begin{description}
\item[@ARR\_WORDS@] is a variable-sized object consisting solely of
non-pointers.  It is used for arrays of all sorts of things (bytes,
words, floats, doubles... it doesn't matter).

Strictly speaking, an @ARR_WORDS@ could be mutable, but because it
only contains non-pointers we don't need to track this fact.

\begin{center}
\begin{tabular}{|c|c|c|c|}
\hline
\emph{Fixed Hdr} & \emph{No of non-pointers} & \emph{Non-pointers\ldots}	\\ \hline
\end{tabular}
\end{center}
\end{description}

\subsubsection{Mutable objects}
\label{sec:mutables}
\label{sec:MUT_VAR}
\label{sec:MUT_ARR_PTRS}
\label{sec:MUT_ARR_PTRS_FROZEN}
\label{sec:MVAR}

Some of these objects are \emph{mutable}; they represent objects which
are explicitly mutated by Haskell code through the @ST@ or @IO@
monads.  They're not used for thunks which are updated precisely once.
Depending on the garbage collector, mutable closures may contain extra
header information which allows a generational collector to implement
the ``write barrier.''

Notice that mutable objects all have the same general layout: there is
a mutable link field as the second word after the header.  This is so
that code to process old-generation mutable lists doesn't need to look
at the type of the object to determine where its link field is.

\begin{description}

\item[@MUT\_VAR@] is a mutable variable.
\begin{center}
\begin{tabular}{|c|c|c|}
\hline
\emph{Fixed Hdr} \emph{Pointer} & \emph{Mutable link} & \\ \hline
\end{tabular}
\end{center}

\item[@MUT\_ARR\_PTRS@] is a mutable array of pointers.  Such an array
may be \emph{frozen}, becoming an @MUT_ARR_PTRS_FROZEN@, with a
different info-table.

\begin{center}
\begin{tabular}{|c|c|c|c|}
\hline
\emph{Fixed Hdr} & \emph{No of ptrs} & \emph{Mutable link} & \emph{Pointers\ldots} \\ \hline
\end{tabular}
\end{center}

\item[@MUT\_ARR\_PTRS\_FROZEN@] This is the immutable version of
@MUT_ARR_PTRS@.  It still has a mutable link field for two reasons: we
need to keep it on the mutable list for an old generation at least
until the next garbage collection, and it may become mutable again via
@thawArray@.

\begin{center}
\begin{tabular}{|c|c|c|c|}
\hline
\emph{Fixed Hdr} & \emph{No of ptrs} & \emph{Mutable link} & \emph{Pointers\ldots} \\ \hline
\end{tabular}
\end{center}

\item[@MVAR@]

\begin{center}
\begin{tabular}{|l|l|l|l|l|}
\hline 
\emph{Fixed header} & \emph{Head} & \emph{Mutable link} & \emph{Tail}
& \emph{Value}\\
\hline
\end{tabular}
\end{center}

\ToDo{MVars}

\end{description}


\Subsubsection{Foreign objects}{FOREIGN}

Here's what a ForeignObj looks like:

\begin{center}
\begin{tabular}{|l|l|l|l|}
\hline 
\emph{Fixed header} & \emph{Data} \\
\hline
\end{tabular}
\end{center}

A foreign object is simple a boxed pointer to an address outside the
Haskell heap, possible to @malloc@ed data.  The only reason foreign
objects exist is so that we can track the lifetime of one using weak
pointers (see \secref{WEAK}) and run a finaliser when the foreign
object is unreachable.

\subsubsection{Weak pointers}
\label{sec:WEAK}

\begin{center}
\begin{tabular}{|l|l|l|l|l|}
\hline 
\emph{Fixed header} & \emph{Key} & \emph{Value} & \emph{Finaliser}
& \emph{Link}\\
\hline
\end{tabular}
\end{center}

\ToDo{Weak poitners}

\subsubsection{Stable names}
\label{sec:STABLE_NAME}

\begin{center}
\begin{tabular}{|l|l|l|l|}
\hline 
\emph{Fixed header} & \emph{Index} \\
\hline
\end{tabular}
\end{center}

\ToDo{Stable names}

The remaining objects types are all administrative --- none of them
may be entered.

\subsection{Other weird objects}
\label{sec:SPARK}
\label{sec:BLOCKED_FETCH}

\begin{description}
\item[@BlockedFetch@ heap objects (`closures')] (parallel only)

@BlockedFetch@s are inbound fetch messages blocked on local closures.
They arise as entries in a local blocking queue when a fetch has been
received for a local black hole.  When awakened, we look at their
contents to figure out where to send a resume.

A @BlockedFetch@ closure has the form:
\begin{center}
\begin{tabular}{|l|l|l|l|l|l|}\hline
\emph{Fixed header} & link & node & gtid & slot & weight \\ \hline
\end{tabular}
\end{center}

\item[Spark Closures] (parallel only)

Spark closures are used to link together all closures in the spark pool.  When
the current processor is idle, it may choose to speculatively evaluate some of
the closures in the pool.  It may also choose to delete sparks from the pool.
\begin{center}
\begin{tabular}{|l|l|l|l|l|l|}\hline
\emph{Fixed header} & \emph{Spark pool link} & \emph{Sparked closure} \\ \hline
\end{tabular}
\end{center}

\item[Slop Objects]\label{sec:slop-objects}

Slop objects are used to overwrite the end of an updatee if it is
larger than an indirection.  Normal slop objects consist of an info
pointer a size word and a number of slop words.  

\begin{center}
\begin{tabular}{|l|l|l|l|l|l|}\hline
\emph{Info Pointer} & \emph{Size} & \emph{Slop Words} \\ \hline
\end{tabular}
\end{center}

This is too large for single word slop objects which consist of a
single info table.

Note that slop objects only contain an info pointer, not a standard
fixed header.  This doesn't cause problems because slop objects are
always unreachable --- they can only be accessed by linearly scanning
the heap.

\note{Currently we don't use slop objects because the storage manager
isn't reliant on objects being adjacent, but if we move to a ``mostly
copying'' style collector, this will become an issue.}

\end{description}

\Subsection{Thread State Objects (TSOs)}{TSO}

In the multi-threaded system, the state of a suspended thread is
packed up into a Thread State Object (TSO) which contains all the
information needed to restart the thread and for the garbage collector
to find all reachable objects.  When a thread is running, it may be
``unpacked'' into machine registers and various other memory locations
to provide faster access.

Single-threaded systems don't really \emph{need\/} TSOs --- but they do
need some way to tell the storage manager about live roots so it is
convenient to use a single TSO to store the mutator state even in
single-threaded systems.

Rather than manage TSOs' alloc/dealloc, etc., in some \emph{ad hoc}
way, we instead alloc/dealloc/etc them in the heap; then we can use
all the standard garbage-collection/fetching/flushing/etc machinery on
them.  So that's why TSOs are ``heap objects,'' albeit very special
ones.
\begin{center}
\begin{tabular}{|l|l|}
   \hline \emph{Fixed header}
\\ \hline \emph{Link field}
\\ \hline \emph{Mutable link field}
\\ \hline \emph{What next}
\\ \hline \emph{State}
\\ \hline \emph{Thread Id}
\\ \hline \emph{Exception Handlers}
\\ \hline \emph{Ticky Info}
\\ \hline \emph{Profiling Info}
\\ \hline \emph{Parallel Info}
\\ \hline \emph{GranSim Info}
\\ \hline \emph{Stack size}
\\ \hline \emph{Max Stack size}
\\ \hline \emph{Sp}
\\ \hline \emph{Su}
\\ \hline \emph{SpLim}
\\ \hline 
\\
          \emph{Stack}
\\
\\ \hline 
\end{tabular}
\end{center}
The contents of a TSO are:
\begin{description}

\item[\emph{Link field}] This is a pointer used to maintain a list of
threads with a similar state (e.g.~all runnable, all sleeping, all
blocked on the same black hole, all blocked on the same MVar,
etc.)

\item[\emph{Mutable link field}] Because the stack is mutable by
definition, the generational collector needs to track TSOs in older
generations that may point into younger ones (which is just about any
TSO for a thread that has run recently).  Hence the need for a mutable
link field (see \secref{mutables}).

\item[\emph{What next}]
This field has five values:  
\begin{description}
\item[@ThreadEnterGHC@]  The thread can be started by entering the
closure pointed to by the word on the top of the stack.
\item[@ThreadRunGHC@]  The thread can be started by jumping to the
address on the top of the stack.
\item[@ThreadEnterHugs@]  The stack has a pointer to a Hugs-built
closure on top of the stack: enter the closure to run the thread.
\item[@ThreadKilled@] The thread has been killed (by @killThread#@).
It is probably still around because it is on some queue somewhere and
hasn't been garbage collected yet.
\item[@ThreadComplete@] The thread has finished.  Its @TSO@ hasn't
been garbage collected yet.
\end{description}

\item[\emph{Thread Id}]
This field contains a (not necessarily unique) integer that identifies
the thread.  It can be used eg. for hashing.

\item[\emph{Ticky Info}] Optional information for ``Ticky Ticky''
statistics: @TSO_STK_HWM@ is the maximum number of words allocated to
this thread.

\item[\emph{Profiling Info}] Optional information for profiling:
@TSO_CCC@ is the current cost centre.

\item[\emph{Parallel Info}]
Optional information for parallel execution.

% \begin{itemize}
% 
% \item The types of threads (@TSO_TYPE@):
% \begin{description}
% \item[@T_MAIN@]     Must be executed locally.
% \item[@T_REQUIRED@] A required thread  -- may be exported.
% \item[@T_ADVISORY@] An advisory thread -- may be exported.
% \item[@T_FAIL@]     A failure thread   -- may be exported.
% \end{description}
% 
% \item I've no idea what else
% 
% \end{itemize}

\item[\emph{GranSim Info}]
Optional information for gransim execution.

% \item Optional information for GranSim execution:
% \begin{itemize}
% \item locked         
% \item sparkname	 
% \item started at	 
% \item exported	 
% \item basic blocks	 
% \item allocs	 
% \item exectime	 
% \item fetchtime	 
% \item fetchcount	 
% \item blocktime	 
% \item blockcount	 
% \item global sparks	 
% \item local sparks	 
% \item queue		 
% \item priority	 
% \item clock          (gransim light only)
% \end{itemize}
% 
% 
% Here are the various queues for GrAnSim-type events.
% 
% Q_RUNNING   
% Q_RUNNABLE  
% Q_BLOCKED   
% Q_FETCHING  
% Q_MIGRATING 
% 

\item[\emph{Stack Info}] Various fields contain information on the
stack: its current size, its maximum size (to avoid infinite loops
overflowing the memory), the current stack pointer (\emph{Sp}), the
current stack update frame pointer (\emph{Su}), and the stack limit
(\emph{SpLim}).  The latter three fields are loaded into the relevant
registers when the thread is run.

\item[\emph{Stack}] This is the actual stack for the thread,
\emph{Stack size} words long.  It grows downwards from higher
addresses to lower addresses.  When the stack overflows, it will
generally be relocated into larger premises unless \emph{Max stack
size} is reached.

\end{description}

The garbage collector needs to be able to find all the
pointers in a stack.  How does it do this?

\begin{itemize}

\item Within the stack there are return addresses, pushed
by @case@ expressions.  Below a return address (i.e. at higher
memory addresses, since the stack grows downwards) is a chunk
of stack that the return address ``knows about'', namely the
activation record of the currently running function.

\item Below each such activation record is a \emph{pending-argument
section}, a chunk of
zero or more words that are the arguments to which the result
of the function should be applied.  The return address does not
statically
``know'' how many pending arguments there are, or their types.
(For example, the function might return a result of type $\alpha$.)

\item Below each pending-argument section is another return address,
and so on.  Actually, there might be an update frame instead, but we
can consider update frames as a special case of a return address with
a well-defined activation record.

\end{itemize}

The game plan is this.  The garbage collector walks the stack from the
top, traversing pending-argument sections and activation records
alternately.  Next we discuss how it finds the pointers in each of
these two stack regions.


\Subsubsection{Activation records}{activation-records}

An \emph{activation record} is a contiguous chunk of stack,
with a return address as its first word, followed by as many
data words as the return address ``knows about''.  The return
address is actually a fully-fledged info pointer.  It points
to an info table, replete with:

\begin{itemize}
\item entry code (i.e. the code to return to).

\item closure type is either @RET_SMALL/RET_VEC_SMALL@ or
@RET_BIG/RET_VEC_BIG@, depending on whether the activation record has
more than 32 data words (\note{64 for 8-byte-word architectures}) and
on whether to use a direct or a vectored return.

\item the layout info for @RET_SMALL@ is a bitmap telling the layout
of the activation record, one bit per word.  The least-significant bit
describes the first data word of the record (adjacent to the fixed
header) and so on.  A ``@1@'' indicates a non-pointer, a ``@0@''
indicates a pointer.  We don't need to indicate exactly how many words
there are, because when we get to all zeros we can treat the rest of
the activation record as part of the next pending-argument region.

For @RET_BIG@ the layout field points to a block of bitmap words,
starting with a word that tells how many words are in the block.

\item the info table contains a Static Reference Table pointer for the
return address (\secref{srt}).
\end{itemize}

The activation record is a fully fledged closure too.  As well as an
info pointer, it has all the other attributes of a fixed header
(\secref{fixed-header}) including a saved cost centre which
is reloaded when the return address is entered.

In other words, all the attributes of closures are needed for
activation records, so it's very convenient to make them look alike.


\Subsubsection{Pending arguments}{pending-args}

So that the garbage collector can correctly identify pointers in
pending-argument sections we explicitly tag all non-pointers.  Every
non-pointer in a pending-argument section is preceded (at the next
lower memory word) by a one-word byte count that says how many bytes
to skip over (excluding the tag word).

The garbage collector traverses a pending argument section from the
top (i.e. lowest memory address).  It looks at each word in turn:

\begin{itemize}
\item If it is less than or equal to a small constant @ARGTAG_MAX@
then it treats it as a tag heralding zero or more words of
non-pointers, so it just skips over them.

\item If it points to the code segment, it must be a return
address, so we have come to the end of the pending-argument section.

\item Otherwise it must be a bona fide heap pointer.
\end{itemize}


\Subsection{The Stable Pointer Table}{STABLEPTR_TABLE}

A stable pointer is a name for a Haskell object which can be passed to
the external world.  It is ``stable'' in the sense that the name does
not change when the Haskell garbage collector runs---in contrast to
the address of the object which may well change.

A stable pointer is represented by an index into the
@StablePointerTable@.  The Haskell garbage collector treats the
@StablePointerTable@ as a source of roots for GC.

In order to provide efficient access to stable pointers and to be able
to cope with any number of stable pointers (eg $0 \ldots 100000$), the
table of stable pointers is an array stored on the heap and can grow
when it overflows.  (Since we cannot compact the table by moving
stable pointers about, it seems unlikely that a half-empty table can
be reduced in size---this could be fixed if necessary by using a
hash table of some sort.)

In general a stable pointer table closure looks like this:

\begin{center}
\begin{tabular}{|l|l|l|l|l|l|l|l|l|l|l|}
\hline
\emph{Fixed header} & \emph{No of pointers} & \emph{Free} & $SP_0$ & \ldots & $SP_{n-1}$ 
\\\hline
\end{tabular}
\end{center}

The fields are:
\begin{description}

\item[@NPtrs@:] number of (stable) pointers.

\item[@Free@:] the byte offset (from the first byte of the object) of the first free stable pointer.

\item[$SP_i$:] A stable pointer slot.  If this entry is in use, it is
an ``unstable'' pointer to a closure.  If this entry is not in use, it
is a byte offset of the next free stable pointer slot.

\end{description}

When a stable pointer table is evacuated
\begin{enumerate}
\item the free list entries are all set to @NULL@ so that the evacuation
  code knows they're not pointers;

\item The stable pointer slots are scanned linearly: non-@NULL@ slots
are evacuated and @NULL@-values are chained together to form a new free list.
\end{enumerate}

There's no need to link the stable pointer table onto the mutable
list because we always treat it as a root.

%%%%%%%%%%%%%%%%%%%%%%%%%%%%%%%%%%%%%%%%%%%%%%%%%%%%%%%%%%%%%%%%
\Subsection{Garbage Collecting CAFs}{CAF}
%%%%%%%%%%%%%%%%%%%%%%%%%%%%%%%%%%%%%%%%%%%%%%%%%%%%%%%%%%%%%%%%

% begin{direct quote from current paper}
A CAF (constant applicative form) is a top-level expression with no
arguments.  The expression may need a large, even unbounded, amount of
storage when it is fully evaluated.

CAFs are represented by closures in static memory that are updated
with indirections to objects in the heap space once the expression is
evaluated.  Previous version of GHC maintained a list of all evaluated
CAFs and traversed them during GC, the result being that the storage
allocated by a CAF would reside in the heap until the program ended.
% end{direct quote from current paper}

% begin{elaboration on why CAFs are very very bad}
Treating CAFs this way has two problems:
\begin{itemize}
\item
It can cause a very large space leak.  For example, this program
should run in constant space but, instead, will run out of memory.
\begin{verbatim}
> main :: IO ()
> main = print nats
>
> nats :: [Int]
> nats = [0..maxInt]
\end{verbatim}

\item
Expressions with no arguments have very different space behaviour
depending on whether or not they occur at the top level.  For example, 
if we make \verb+nats+ a local definition, the space leak goes away 
and the resulting program runs in constant space, as expected.
\begin{verbatim}
> main :: IO ()
> main = print nats
>  where
>   nats :: [Int]
>   nats = [0..maxInt]
\end{verbatim}

This huge change in the operational behaviour of the program 
is a problem for optimising compilers and for programmers.
For example, GHC will normally flatten a set of let bindings using
this transformation:
\begin{verbatim}
let x1 = let x2 = e2 in e1   ==>   let x2 = e2 in let x1 = e1
\end{verbatim}
but it does not do so if this would raise \verb+x2+ to the top level
since that may create a CAF.  Many Haskell programmers avoid creating
large CAFs by adding a dummy argument to a CAF or by moving a CAF away
from the top level.

\end{itemize}
% end{elaboration on why CAFs are very very bad}

Solving the CAF problem requires different treatment in interactive
systems such as Hugs than in batch-mode systems such as GHC 
\begin{itemize}
\item
In a batch-mode the program the runtime system is terminated
after every execution of the runtime system.  In such systems,
the garbage collector can completely ``destroy'' a CAF when it 
is no longer live --- in much the same way as it ``destroys''
normal closures when they are no longer live.

\item
In an interactive system, many expressions are evaluated without
restarting the runtime system between each evaluation.  In such
systems, the garbage collector cannot completely ``destroy'' a CAF
when it is no longer live because, whilst it might not be required in
the evaluation of the current expression, it might be required in the
next evaluation.

There are two possible behaviours we might want:
\begin{enumerate}
\item
When a CAF is no longer required for the current evaluation, the CAF
should be reverted to its original form.  This behaviour ensures that
the operational behaviour of the interactive system is a reasonable
predictor of the operational behaviour of the batch-mode system.  This
allows us to use Hugs for performance debugging (in particular, trying
to understand and reduce the heap usage of a program) --- an area of
increasing importance as Haskell is used more and more to solve ``real
problems'' in ``real problem domains''.

\item
Even if a CAF is no longer required for the current evaluation, we might
choose to hang onto it by collecting it in the normal way.  This keeps
the space leak but might be useful in a teaching environment when
trying to teach the difference between call by name evaluation (which
doesn't share work) and lazy evaluation (which does share work).

\end{enumerate}

It turns out that it is easy to support both styles of use, so the
runtime system provides a switch which lets us turn this on and off
during execution.  \ToDo{What is this switch called?}  It would also
be easy to provide a function \verb+RevertCAF+ to let the interpreter
revert any CAF it wanted between (but not during) executions, if we so
desired.  Running \verb+RevertCAF+ during execution would lose some sharing
but is otherwise harmless.

\end{itemize}

% % begin{even more pointless observation?}
% The simplest fix would be to remove the special treatment of 
% top level variables.  This works but is very inefficient.
% ToDo: say why.
% (Note: delete this paragraph from final version.)
% % end{even more pointless observation?}

% begin{pointless observation?}
An easy but inefficient fix to the CAF problem would be to make a
complete copy of the heap before every evaluation and discard the copy
after evaluation.  This works but is inefficient.
% end{pointless observation?}

An efficient way to achieve a similar effect is to revert all
updatable thunks to their original form as they become unnecessary for
the current evaluation.  To do this, we modify the compiler to ensure
that the only updatable thunks generated by the compiler are CAFs and
we modify the garbage collector to revert entered CAFs to unentered
CAFs as their value becomes unnecessary.


\subsubsection{New Heap Objects}

We add three new kinds of heap object: unentered CAF closures, entered
CAF objects and CAF blackholes.  We first describe how they are
evaluated and then how they are garbage collected.
\begin{itemize}
\item
Unentered CAF closures contain a pointer to closure representing the
body of the CAF.  The ``body closure'' is not updatable.

Unentered CAF closures contain two unused fields to make them the same
size as entered CAF closures --- which allows us to perform an inplace
update.  \ToDo{Do we have to add another kind of inplace update operation
to the storage manager interface or do we consider this to be internal
to the SM?}
\begin{center}
\begin{tabular}{|l|l|l|l|}\hline
\verb+CAF_unentered+ & \emph{body closure} & \emph{unused} & \emph{unused} \\ \hline
\end{tabular}
\end{center}
When an unentered CAF is entered, we do the following:
\begin{itemize}
\item
allocate a CAF black hole;

\item
push an update frame (to update the CAF black hole) onto the stack;

\item
overwrite the CAF with an entered CAF object (see below) with the same
body and whose value field points to the black hole;

\item
add the CAF to a list of all entered CAFs (called ``the CAF list'');
and

\item
the closure representing the value of the CAF is entered.

\end{itemize}

When evaluation of the CAF body returns a value, the update frame
causes the CAF black hole to be updated with the value in the normal
way.

\ToDo{Add a picture}

\item
Entered CAF closures contain two pointers: a pointer to the CAF body
(the same as for unentered CAF closures); a pointer to the CAF value
(this is initialised with a CAF blackhole, as previously described);
and a link to the next CAF in the CAF list 

\ToDo{How is the end of the list marked?  Null pointer or sentinel value?}.

\begin{center}
\begin{tabular}{|l|l|l|l|}\hline
\verb+CAF_entered+ & \emph{body closure} & \emph{value} & \emph{link} \\ \hline
\end{tabular}
\end{center}
When an entered CAF is entered, it enters its value closure.

\item
CAF blackholes are identical to normal blackholes except that they
have a different infotable.  The only reason for having CAF blackholes
is to allow an optimisation of lazy blackholing where we stop scanning
the stack when we see the first {\em normal blackhole} but not
when we see a {\em CAF blackhole.}
\ToDo{The optimisation we want to allow should be described elsewhere
so that all we have to do here is describe the difference.}

Instead of allocating a blackhole to update with the value of the CAF,
it might seem simpler to update the CAF directly.  This would require
a new kind of update frame which would update the value field of the
CAF with a pointer to the value and wouldn't catch blackholes caused
by CAFs that depend on themselves so we chose not to do so.

\end{itemize}

\subsubsection{Garbage Collection}

To avoid the space leak, each run of the garbage collector must revert
the entered CAFs which are not required to complete the current
evaluation (that is all the closures reachable from the set of
runnable threads and the stable pointer table).

It does this by performing garbage collection in three phases:
\begin{enumerate}
\item
During the first phase, we ``mark'' all closures reachable from the
scheduler state.  

How we ``mark'' closures depends on the garbage collector.  For
example, in a 2-space collector, closures are ``marked'' by copying
them into ``to-space'', overwriting them with a forwarding node and
``marking'' all the closures reachable from the copy.  The only
requirements are that we can test whether a closure is marked and if a
closure is marked then so are all closures reachable from it.

\ToDo{At present we say that the scheduler state includes any state
that Hugs may have.  This is not true anymore.}

Performing this phase first provides us with a cheap test for
execution closures: at this stage in execution, the execution closures
are precisely the marked closures.

\item
During the second phase, we revert all unmarked CAFs on the CAF list
and remove them from the CAF list.

Since the CAF list is exactly the set of all entered CAFs, this reverts
all entered CAFs which are not execution closures.

\item
During the third phase, we mark all top level objects (including CAFs)
by calling \verb+MarkHugsRoots+ which will call \verb+MarkRoot+ for
each top level object known to Hugs.

\end{enumerate}

To implement the second style of interactive behaviour (where we
deliberately keep the CAF-related space leak), we simply omit the
second phase.  Omitting the second phase causes the third phase to
mark any unmarked CAF value closures.

So far, we have been describing a pure Hugs system which contains no
machine generated code.  The main difference in a hybrid system is
that GHC-generated code is statically allocated in memory instead of
being dynamically allocated on the heap.  We split both
\verb+CAF_unentered+ and \verb+CAF_entered+ into two versions: a
static and a dynamic version.  The static and dynamic versions of each
CAF differ only in whether they are moved during garbage collection.
When reverting CAFs, we revert dynamic entered CAFs to dynamic
unentered CAFs and static entered CAFs to static unentered CAFs.




\Section{The Bytecode Evaluator}{bytecode-evaluator}

This section describes how the Hugs interpreter interprets code in the
same environment as compiled code executes.  Both evaluation models
use a common garbage collector, so they must agree on the form of
objects in the heap.

Hugs interprets code by converting it to byte-code and applying a
byte-code interpreter to it.  Wherever possible, we try to ensure that
the byte-code is all that is required to interpret a section of code.
This means not dynamically generating info tables, and hence we can
only have a small number of possible heap objects each with a statically
compiled info table.  Similarly for stack objects: in fact we only
have one Hugs stack object, in which all information is tagged for the
garbage collector.

There is, however, one exception to this rule.  Hugs must generate
info tables for any constructors it is asked to compile, since the
alternative is to force a context-switch each time compiled code
enters a Hugs-built constructor, which would be prohibitively
expensive.

We achieve this simplicity by forgoing some of the optimisations used
by compiled code:
\begin{itemize}
\item

Whereas compiled code has five different ways of entering a closure
(\secref{ghc-fun-call}), interpreted code has only one.
The entry point for interpreted code behaves like slow entry points for
compiled code.

\item

We use just one info table for \emph{all\/} direct returns.  
This introduces two problems:
\begin{enumerate}
\item How does the interpreter know what code to execute?

Instead of pushing just a return address, we push a return BCO and a 
trivial return address which just enters the return BCO.

(In a purely interpreted system, we could avoid pushing the trivial
return address.)

\item How can the garbage collector follow pointers within the
activation record?

We could push a third word ---a bitmask describing the location of any
pointers within the record--- but, since we're already tagging unboxed
function arguments on the stack, we use the same mechanism for unboxed
values within the activation record.

\ToDo{Do we have to stub out dead variables in the activation frame?}

\end{enumerate}

\item

We trivially support vectored returns by pushing a return vector whose
entries are all the same.

\item

We avoid the need to build SRTs by putting bytecode objects on the
heap and restricting BCOs to a single basic block.

\end{itemize}

\Subsection{Hugs Info Tables}{hugs-info-tables}

Hugs requires the following info tables and closures:
\begin{description}
\item [@HUGS\_RET@].

Contains both a vectored return table and a direct entry point.  All
entry points are the same: they rearrange the stack to match the Hugs
return convention (\secref{hugs-return-convention}) and return to the
scheduler.  When the scheduler restarts the thread, it will find a BCO
on top of the stack and will enter the Hugs interpreter.

\item [@UPD\_RET@].

This is just the standard info table for an update frame.

\item [Constructors].

The entry code for a constructor jumps to a generic entry point in the
runtime system which decides whether to do a vectored or unvectored
return depending on the shape of the constructor/type.  This implies that
info tables must have enough info to make that decision.

\item [@AP@ and @PAP@].

\item [Indirections].

\item [Selectors].

Hugs doesn't generate them itself but it ought to recognise them

\item [Complex primops].

Some of the primops are too complex for GHC to generate inline.
Instead, these primops are hand-written and called as normal functions.
Hugs only needs to know their names and types but doesn't care whether
they are generated by GHC or by hand.  Two things to watch:

\begin{enumerate}
\item
Hugs must be able to enter these primops even if it is working on a
standalone system that does not support genuine GHC generated code.

\item The complex primops often involve unboxed tuple types (which
Hugs does not support at the source level) so we cannot specify their
types in a Haskell source file.

\end{enumerate}

\end{description}

\Subsection{Hugs Heap Objects}{hugs-heap-objects}

\subsubsection{Byte-code objects}

Compiled byte code lives on the global heap, in objects called
Byte-Code Objects (or BCOs).  The layout of BCOs is described in
detail in \secref{BCO}, in this section we will describe
their semantics.

Since byte-code lives on the heap, it can be garbage collected just
like any other heap-resident data.  Hugs arranges that any BCO's
referred to by the Hugs symbol tables are treated as live objects by
the garbage collector.  When a module is unloaded, the pointers to its
BCOs are removed from the symbol table, and the code will be garbage
collected some time later.

A BCO represents a basic block of code --- the (only) entry points is
at the beginning of a BCO, and it is impossible to jump into the
middle of one.  A BCO represents not only the code for a function, but
also its closure; a BCO can be entered just like any other closure.
Hugs performs lambda-lifting during compilation to byte-code, and each
top-level combinator becomes a BCO in the heap.


\subsubsection{Thunks and partial applications}

A thunk consists of a code pointer, and values for the free variables
of that code.  Since Hugs byte-code is lambda-lifted, free variables
become arguments and are expected to be on the stack by the called
function.

Hugs represents updateable thunks with @AP_UPD@ objects applying a closure
to a list of arguments.  (As for @PAP@s, unboxed arguments should be
preceded by a tag.)  When it is entered, it pushes an update frame
followed by its payload on the stack, and enters the first word (which
will be a pointer to a BCO).  The layout of @AP_UPD@ objects is described
in more detail in \secref{AP_UPD}.

Partial applications are represented by @PAP@ objects, which are
non-updatable.

\ToDo{Hugs Constructors}.

\Subsection{Calling conventions}{hugs-calling-conventions}

The calling convention for any byte-code function is straightforward:
\begin{itemize}
\item Push any arguments on the stack.
\item Push a pointer to the BCO.
\item Begin interpreting the byte code.
\end{itemize}

In a system containing both GHC and Hugs, the bytecode interpreter
only has to be able to enter BCOs: everything else can be handled by
returning to the compiled world (as described in
\secref{hugs-to-ghc-switch}) and entering the closure
there.

This would work but it would obviously be very inefficient if we
entered a @AP@ by switching worlds, entering the @AP@, pushing the
arguments and function onto the stack, and entering the function
which, likely as not, will be a byte-code object which we will enter
by \emph{returning} to the byte-code interpreter.  To avoid such
gratuitious world switching, we choose to recognise certain closure
types as being ``standard'' --- and duplicate the entry code for the
``standard closures'' in the bytecode interpreter.

A closure is said to be ``standard'' if its entry code is entirely
determined by its info table.  \emph{Standard Closures} have the
desirable property that the byte-code interpreter can enter the
closure by simply ``interpreting'' the info table instead of switching
to the compiled world.  The standard closures include:

\begin{description}
\item[Constructor] To enter a constructor, we simply return (see
\secref{hugs-return-convention}).

\item[Indirection]
To enter an indirection, we simply enter the object it points to
after possibly adjusting the current cost centre.

\item[@AP@] 

To enter an @AP@, we push an update frame, push the
arguments, push the function and enter the function.
(Not forgetting a stack check at the start.)

\item[@PAP@]

To enter a @PAP@, we push the arguments, push the function and enter
the function.  (Not forgetting a stack check at the start.)

\item[Selector]

To enter a selector (\secref{THUNK_SELECTOR}), we test whether the
selectee is a value.  If so, we simply select the appropriate
component; if not, it's simplest to treat it as a GHC-built closure
--- though we could interpret it if we wanted.

\end{description}

The most obvious omissions from the above list are @BCO@s (which we
dealt with above) and GHC-built closures (which are covered in
\secref{hugs-to-ghc-switch}).


\Subsection{Return convention}{hugs-return-convention}

When Hugs pushes a return address, it pushes both a pointer to the BCO
to return to, and a pointer to a static code fragment @HUGS_RET@ (this
is described in \secref{ghc-to-hugs-switch}).  The
stack layout is shown in \figref{hugs-return-stack}.

\begin{figure}[ht]
\begin{center}
\begin{verbatim}
| stack    |
+----------+
| bco      |--> BCO
+----------+
| HUGS_RET |
+----------+
\end{verbatim}
%\input{hugs_ret.pstex_t}
\end{center}
\caption{Stack layout for a Hugs return address}
\label{fig:hugs-return-stack}
% this figure apparently duplicates {fig:hugs-return-stack1} earlier.
\end{figure}

\begin{figure}[ht]
\begin{center}
\begin{verbatim}
| stack    |
+----------+
| con      |--> CON
+----------+
\end{verbatim}
%\input{hugs_ret2.pstex_t}
\end{center}
\caption{Stack layout on enterings a Hugs return address}
\label{fig:hugs-return2}
\end{figure}

\begin{figure}[ht]
\begin{center}
\begin{verbatim}
| stack    |
+----------+
| 3#       |
+----------+
| I#       |
+----------+
\end{verbatim}
%\input{hugs_ret2.pstex_t}
\end{center}
\caption{Stack layout on entering a Hugs return address with an unboxed value}
\label{fig:hugs-return-int1}
\end{figure}

\begin{figure}[ht]
\begin{center}
\begin{verbatim}
| stack    |
+----------+
| ghc_ret  |
+----------+
| con      |--> CON
+----------+
\end{verbatim}
%\input{hugs_ret3.pstex_t}
\end{center}
\caption{Stack layout on enterings a GHC return address}
\label{fig:hugs-return3}
\end{figure}

\begin{figure}[ht]
\begin{center}
\begin{verbatim}
| stack    |
+----------+
| ghc_ret  |
+----------+
| 3#       |
+----------+
| I#       |
+----------+
| restart  |--> id_Int#_closure
+----------+
\end{verbatim}
%\input{hugs_ret2.pstex_t}
\end{center}
\caption{Stack layout on enterings a GHC return address with an unboxed value}
\label{fig:hugs-return-int}
\end{figure}

When a Hugs byte-code sequence enters a closure, it examines the 
return address on top of the stack.

\begin{itemize}

\item If the return address is @HUGS_RET@, pop the @HUGS_RET@ and the
bco for the continuation off the stack, push a pointer to the constructor onto
the stack and enter the BCO with the current object pointer set to the BCO
(\figref{hugs-return2}).

\item If the top of the stack is not @HUGS_RET@, we need to do a world
switch as described in \secref{hugs-to-ghc-switch}.

\end{itemize}

\ToDo{This duplicates what we say about switching worlds
(\secref{switching-worlds}) - kill one or t'other.}


\ToDo{This was in the evaluation model part but it really belongs in
this part which is about the internal details of each of the major
sections.}

\Subsection{Addressing Modes}{hugs-addressing-modes}

To avoid potential alignment problems and simplify garbage collection,
all literal constants are stored in two tables (one boxed, the other
unboxed) within each BCO and are referred to by offsets into the tables.
Slots in the constant tables are word aligned.

\ToDo{How big can the offsets be?  Is the offset specified in the
address field or in the instruction?}

Literals can have the following types: char, int, nat, float, double,
and pointer to boxed object.  There is no real difference between
char, int, nat and float since they all occupy 32 bits --- but it
costs almost nothing to distinguish them and may improve portability
and simplify debugging.

\Subsection{Compilation}{hugs-compilation}


\def\is{\mbox{\it is}}
\def\ts{\mbox{\it ts}}
\def\as{\mbox{\it as}}
\def\bs{\mbox{\it bs}}
\def\cs{\mbox{\it cs}}
\def\rs{\mbox{\it rs}}
\def\us{\mbox{\it us}}
\def\vs{\mbox{\it vs}}
\def\ws{\mbox{\it ws}}
\def\xs{\mbox{\it xs}}

\def\e{\mbox{\it e}}
\def\alts{\mbox{\it alts}}
\def\fail{\mbox{\it fail}}
\def\panic{\mbox{\it panic}}
\def\ua{\mbox{\it ua}}
\def\obj{\mbox{\it obj}}
\def\bco{\mbox{\it bco}}
\def\tag{\mbox{\it tag}}
\def\entry{\mbox{\it entry}}
\def\su{\mbox{\it su}}

\def\Ind#1{{\mbox{\it Ind}\ {#1}}}
\def\update#1{{\mbox{\it update}\ {#1}}}

\def\next{$\Longrightarrow$}
\def\append{\mathrel{+\mkern-6mu+}}
\def\reverse{\mbox{\it reverse}}
\def\size#1{{\vert {#1} \vert}}
\def\arity#1{{\mbox{\it arity}{#1}}}

\def\AP{\mbox{\it AP}}
\def\PAP{\mbox{\it PAP}}
\def\GHCRET{\mbox{\it GHCRET}}
\def\GHCOBJ{\mbox{\it GHCOBJ}}

To make sense of the instructions, we need a sense of how they will be
used.  Here is a small compiler for the STG language.

\begin{verbatim}
> cg (f{a1, ... am}) = do
>   pushAtom am; ... pushAtom a1
>   pushVar f
>   SLIDE (m+1) |env|
>   ENTER
> cg (let {x1=rhs1; ... xm=rhsm} in e) = do
>   ALLOC x1 |rhs1|, ... ALLOC xm |rhsm|
>   build x1 rhs1,   ... build xm rhsm
>   cg e
> cg (case e of alts) = do
>   PUSHALTS (cgAlts alts)
>   cg e

> cgAlts { alt1; ... altm }  = cgAlt alt1 $ ... $ cgAlt altm pmFail
>
> cgAlt (x@C{xs} -> e) fail = do
>   TEST C fail
>   HEAPCHECK (heapUse e)
>   UNPACK xs
>   cg e

> build x (C{a1, ... am}) = do 
>   pushUntaggedAtom am; ... pushUntaggedAtom a1
>   PACK x C
> -- A useful optimisation
> build x ({v1, ... vm} \ {}. f{a1, ... am}) = do 
>   pushVar am; ... pushVar a1
>   pushVar f
>   MKAP x m
> build x ({v1, ... vm} \ {}. e) = do 
>   pushVar vm; ... pushVar v1
>   PUSHBCO (cgRhs ({v1, ... vm} \ {}. e))
>   MKAP x m
> build x ({v1, ... vm} \ {x1, ... xm}. e) = do 
>   pushVar vm; ... pushVar v1
>   PUSHBCO (cgRhs ({v1, ... vm} \ {x1, ... xm}. e))
>   MKPAP x m

> cgRhs (vs \ xs. e) = do
>   ARGCHECK   (xs ++ vs)  -- can be omitted if xs == {}
>   STACKCHECK min(stackUse e,heapOverflowSlop)
>   HEAPCHECK  (heapUse e)
>   cg e

> pushAtom x  = pushVar x
> pushAtom i# = PUSHINT i#

> pushVar x = if isGlobalVar x then PUSHGLOBAL x else PUSHLOCAL x 

> pushUntaggedAtom x  = pushVar x
> pushUntaggedAtom i# = PUSHUNTAGGEDINT i#

> pushVar x = if isGlobalVar x then PUSHGLOBAL x else PUSHLOCAL x 
\end{verbatim}

\ToDo{Is there an easy way to add semi-tagging?  Would it be that different?}

\ToDo{Optimise thunks of the form @f{x1,...xm}@ so that we build an AP directly}

\Subsection{Instructions}{hugs-instructions}

We specify the semantics of instructions using transition rules of
the form:

\begin{tabular}{|llrrrrr|}
\hline
	& $\is$		& $s$ 	& $\su$ 	& $h$  & $hp$  & $\sigma$ \\
\next	& $\is'$	& $s'$ 	& $\su'$	& $h'$ & $hp'$ & $\sigma$ \\
\hline
\end{tabular}

where $\is$ is an instruction stream, $s$ is the stack, $\su$ is the 
update frame pointer and $h$ is the heap.


\Subsection{Stack manipulation}{hugs-stack-manipulation}

\begin{description}

\item[ Push a global variable ].

\begin{tabular}{|llrrrrr|}
\hline
	& PUSHGLOBAL $o$ : $\is$ & $s$ 		& $su$ & $h$ & $hp$ & $\sigma$ \\
\next	& $\is$			 & $\sigma!o:s$ & $su$ & $h$ & $hp$ & $\sigma$ \\
\hline
\end{tabular}

\item[ Push a local variable ].

\begin{tabular}{|llrrrrr|}
\hline
	& PUSHLOCAL $o$ : $\is$	& $s$ 		& $su$ & $h$ & $hp$ & $\sigma$ \\
\next	& $\is$			& $s!o : s$ 	& $su$ & $h$ & $hp$ & $\sigma$ \\
\hline
\end{tabular}

\item[ Push an unboxed int ].

\begin{tabular}{|llrrrrr|}
\hline
	& PUSHINT $o$ : $\is$	& $s$ 		        & $su$ & $h$ & $hp$ & $\sigma$ \\
\next	& $\is$			& $I\# : \sigma!o : s$ 	& $su$ & $h$ & $hp$ & $\sigma$ \\
\hline
\end{tabular}

The $I\#$ is a tag included for the benefit of the garbage collector.
Similar rules exist for floats, doubles, chars, etc.

\item[ Push an unboxed int ].

\begin{tabular}{|llrrrrr|}
\hline
	& PUSHUNTAGGEDINT $o$ : $\is$	& $s$ 		        & $su$ & $h$ & $hp$ & $\sigma$ \\
\next	& $\is$			& $\sigma!o : s$ 	& $su$ & $h$ & $hp$ & $\sigma$ \\
\hline
\end{tabular}

Similar rules exist for floats, doubles, chars, etc.

\item[ Delete environment from stack --- ready for tail call ].

\begin{tabular}{|llrrrrr|}
\hline
	& SLIDE $m$ $n$ : $\is$	& $\as \append \bs \append \cs$		& $su$ & $h$ & $hp$ & $\sigma$ \\
\next	& $\is$			& $\as \append \cs$			& $su$ & $h$ & $hp$ & $\sigma$ \\
\hline
\end{tabular}
\\
where $\size{\as} = m$ and $\size{\bs} = n$.


\item[ Push a return address ].

\begin{tabular}{|llrrrrr|}
\hline
	& PUSHALTS $o$:$\is$	& $s$ 			& $su$ & $h$ & $hp$ & $\sigma$ \\
\next	& $\is$			& $@HUGS_RET@:\sigma!o:s$ 	& $su$ & $h$ & $hp$ & $\sigma$ \\
\hline
\end{tabular}

\item[ Push a BCO ].

\begin{tabular}{|llrrrrr|}
\hline
	& PUSHBCO $o$ : $\is$	& $s$ 			& $su$ & $h$ & $hp$ & $\sigma$ \\
\next	& $\is$			& $\sigma!o : s$ 	& $su$ & $h$ & $hp$ & $\sigma$ \\
\hline
\end{tabular}

\end{description}

%%%%%%%%%%%%%%%%%%%%%%%%%%%%%%%%%%%%%%%%%%%%%%%%%%%%%%%%%%%%%%%%
\Subsection{Heap manipulation}{hugs-heap-manipulation}
%%%%%%%%%%%%%%%%%%%%%%%%%%%%%%%%%%%%%%%%%%%%%%%%%%%%%%%%%%%%%%%%

\begin{description}

\item[ Allocate a heap object ].

\begin{tabular}{|llrrrrr|}
\hline
	& ALLOC $m$ : $\is$	& $s$    & $su$ & $h$ & $hp$   & $\sigma$ \\
\next	& $\is$			& $hp:s$ & $su$ & $h$ & $hp+m$ & $\sigma$ \\
\hline
\end{tabular}

\item[ Build a constructor ].

\begin{tabular}{|llrrrrr|}
\hline
	& PACK $o$ $o'$ : $\is$	& $\ws \append s$ 	& $su$ & $h$ 				& $hp$ & $\sigma$ \\
\next	& $\is$			& $s$ 			& $su$ & $h[s!o \mapsto Pack C\{\ws\}]$	& $hp$ & $\sigma$ \\
\hline
\end{tabular}
\\
where $C = \sigma!o'$ and $\size{\ws} = \arity{C}$.

\item[ Build an AP or  PAP ].

\begin{tabular}{|llrrrrr|}
\hline
	& MKAP $o$ $m$:$\is$	& $f : \ws \append s$	& $su$ & $h$ 				& $hp$ & $\sigma$ \\
\next	& $\is$			& $s$ 			& $su$ & $h[s!o \mapsto \AP(f,\ws)]$ 	& $hp$ & $\sigma$ \\
\hline
\end{tabular}
\\
where $\size{\ws} = m$.

\begin{tabular}{|llrrrrr|}
\hline
	& MKPAP $o$ $m$:$\is$	& $f : \ws \append s$	& $su$ & $h$ 				& $hp$ & $\sigma$ \\
\next	& $\is$			& $s$ 			& $su$ & $h[s!o \mapsto \PAP(f,\ws)]$ 	& $hp$ & $\sigma$ \\
\hline
\end{tabular}
\\
where $\size{\ws} = m$.

\item[ Unpacking a constructor ].

\begin{tabular}{|llrrrrr|}
\hline
	& UNPACK : $is$ 	& $a : s$ 				& $su$ & $h[a \mapsto C\ \ws]$  	& $hp$ & $\sigma$ \\
\next	& $is'$			& $(\reverse\ \ws) \append a : s$ 	& $su$ & $h$ 				& $hp$ & $\sigma$ \\
\hline
\end{tabular}

The $\reverse\ \ws$ looks expensive but, since the stack grows down
and the heap grows up, that's actually the cheap way of copying from
heap to stack.  Looking at the compilation rules, you'll see that we
always push the args in reverse order.

\end{description}


\Subsection{Entering a closure}{hugs-entering}

\begin{description}

\item[ Enter a BCO ].

\begin{tabular}{|llrrrrr|}
\hline
	& [ENTER]	& $a : s$ 	& $su$ & $h[a \mapsto BCO\{\is\} ]$  	& $hp$ & $\sigma$ \\
\next	& $\is$ 	& $a : s$ 	& $su$ & $h$ 				& $hp$ & $a$ \\
\hline
\end{tabular}

\item[ Enter a PAP closure ].

\begin{tabular}{|llrrrrr|}
\hline
	& [ENTER]	& $a : s$ 		& $su$ & $h[a \mapsto \PAP(f,\ws)]$  	& $hp$ & $\sigma$ \\
\next	& [ENTER] 	& $f : \ws \append s$ 	& $su$ & $h$ 				& $hp$ & $???$ \\
\hline
\end{tabular}

\item[ Entering an AP closure ].

\begin{tabular}{|llrrrrr|}
\hline
	& [ENTER]	& $a : s$ 				& $su$ 	& $h[a \mapsto \AP(f,ws)]$  	& $hp$ & $\sigma$ \\
\next	& [ENTER] 	& $f : \ws \append @UPD_RET@:\su:a:s$ 	& $su'$	& $h$ 				& $hp$ & $???$ \\
\hline
\end{tabular}

Optimisations:
\begin{itemize}
\item Instead of blindly pushing an update frame for $a$, we can first test whether there's already
 an update frame there.  If so, overwrite the existing updatee with an indirection to $a$ and
 overwrite the updatee field with $a$.  (Overwriting $a$ with an indirection to the updatee also
 works.)  This results in update chains of maximum length 2. 
\end{itemize}


\item[ Returning a constructor ].

\begin{tabular}{|llrrrrr|}
\hline
	& [ENTER]		& $a : @HUGS_RET@ : \alts : s$ 	& $su$ & $h[a \mapsto C\{\ws\}]$  	& $hp$ & $\sigma$ \\
\next	& $\alts.\entry$	& $a:s$ 			& $su$ & $h$ 				& $hp$ & $\sigma$ \\
\hline
\end{tabular}


\item[ Entering an indirection node ].

\begin{tabular}{|llrrrrr|}
\hline
	& [ENTER]	& $a  : s$ 	& $su$ & $h[a \mapsto \Ind{a'}]$ 	& $hp$ & $\sigma$ \\
\next	& [ENTER]	& $a' : s$ 	& $su$ & $h$ 				& $hp$ & $\sigma$ \\
\hline
\end{tabular}

\item[Entering GHC closure].

\begin{tabular}{|llrrrrr|}
\hline
	& [ENTER]	& $a : s$ 	& $su$ & $h[a \mapsto \GHCOBJ]$  	& $hp$ & $\sigma$ \\
\next	& [ENTERGHC] 	& $a : s$ 	& $su$ & $h$ 				& $hp$ & $\sigma$ \\
\hline
\end{tabular}

\item[Returning a constructor to GHC].

\begin{tabular}{|llrrrrr|}
\hline
	& [ENTER]	& $a : \GHCRET : s$ 	& $su$ & $h[a \mapsto C \ws]$  	& $hp$ & $\sigma$ \\
\next	& [ENTERGHC]	& $a : \GHCRET : s$ 	& $su$ & $h$ 			& $hp$ & $\sigma$ \\
\hline
\end{tabular}

\end{description}


\Subsection{Updates}{hugs-updates}

\begin{description}

\item[ Updating with a constructor].

\begin{tabular}{|llrrrrr|}
\hline
	& [ENTER]	& $a : @UPD_RET@ : ua : s$ 	& $su$ & $h[a \mapsto C\{\ws\}]$  & $hp$ & $\sigma$ \\
\next	& [ENTER]	& $a \append s$ 		& $su$ & $h[au \mapsto \Ind{a}$   & $hp$ & $\sigma$ \\
\hline
\end{tabular}

\item[ Argument checks].

\begin{tabular}{|llrrrrr|}
\hline
	& ARGCHECK $m$:$\is$	& $a : \as \append s$ 	& $su$ & $h$ 	& $hp$ & $\sigma$ \\
\next	& $\is$			& $a : \as \append s$ 	& $su$ & $h'$ 	& $hp$ & $\sigma$ \\
\hline
\end{tabular}
\\
where $m \ge (su - sp)$

\begin{tabular}{|llrrrrr|}
\hline
	& ARGCHECK $m$:$\is$	& $a : \as \append @UPD_RET@:su:ua:s$ 	& $su$ & $h$ 	& $hp$ & $\sigma$ \\
\next	& $\is$			& $a : \as \append s$ 			& $su$ & $h'$ 	& $hp$ & $\sigma$ \\
\hline
\end{tabular}
\\
where $m < (su - sp)$ and
      $h' = h[ua \mapsto \Ind{a'}, a' \mapsto \PAP(a,\reverse\ \as) ]$

Again, we reverse the list of values as we transfer them from the
stack to the heap --- reflecting the fact that the stack and heap grow
in different directions.

\end{description}

\Subsection{Branches}{hugs-branches}

\begin{description}

\item[ Testing a constructor ].

\begin{tabular}{|llrrrrr|}
\hline
	& TEST $tag$ $is'$ : $is$ 	& $a : s$ 	& $su$ & $h[a \mapsto C\ \ws]$ 	& $hp$ & $\sigma$ \\
\next	& $is$				& $a : s$ 	& $su$ & $h$ 			& $hp$ & $\sigma$ \\
\hline
\end{tabular}
\\
where $C.\tag = tag$

\begin{tabular}{|llrrrrr|}
\hline
	& TEST $tag$ $is'$ : $is$ 	& $a : s$ 	& $su$ & $h[a \mapsto C\ \ws]$  & $hp$ & $\sigma$ \\
\next	& $is'$				& $a : s$ 	& $su$ & $h$ 			& $hp$ & $\sigma$ \\
\hline
\end{tabular}
\\
where $C.\tag \neq tag$

\end{description}

%%%%%%%%%%%%%%%%%%%%%%%%%%%%%%%%%%%%%%%%%%%%%%%%%%%%%%%%%%%%%%%%
\Subsection{Heap and stack checks}{hugs-heap-stack-checks}
%%%%%%%%%%%%%%%%%%%%%%%%%%%%%%%%%%%%%%%%%%%%%%%%%%%%%%%%%%%%%%%%

\begin{tabular}{|llrrrrr|}
\hline
	& STACKCHECK $stk$:$\is$	& $s$ 	& $su$ & $h$  	& $hp$ & $\sigma$ \\
\next	& $\is$ 			& $s$ 	& $su$ & $h$	& $hp$ & $\sigma$ \\
\hline
\end{tabular}
\\
if $s$ has $stk$ free slots.

\begin{tabular}{|llrrrrr|}
\hline
	& HEAPCHECK $hp$:$\is$		& $s$ 	& $su$ & $h$  	& $hp$ & $\sigma$ \\
\next	& $\is$ 			& $s$ 	& $su$ & $h$	& $hp$ & $\sigma$ \\
\hline
\end{tabular}
\\
if $h$ has $hp$ free slots.

If either check fails, we push the current bco ($\sigma$) onto the
stack and return to the scheduler.  When the scheduler has fixed the
problem, it pops the top object off the stack and reenters it.


Optimisations:
\begin{itemize}
\item The bytecode CHECK1000 conservatively checks for 1000 words of heap space and 1000 words of stack space.
      We use it to reduce code space and instruction decoding time.
\item The bytecode HEAPCHECK1000 conservatively checks for 1000 words of heap space.
      It is used in case alternatives.
\end{itemize}


%%%%%%%%%%%%%%%%%%%%%%%%%%%%%%%%%%%%%%%%%%%%%%%%%%%%%%%%%%%%%%%%
\Subsection{Primops}{hugs-primops}
%%%%%%%%%%%%%%%%%%%%%%%%%%%%%%%%%%%%%%%%%%%%%%%%%%%%%%%%%%%%%%%%

\ToDo{primops take m words and return n words. The expect boxed arguments on the stack.}


\Section{The Machine Code Evaluator}{asm-evaluator}

This section describes the framework in which compiled code evaluates
expressions.  Only at certain points will compiled code need to be
able to talk to the interpreted world; these are discussed in
\secref{switching-worlds}.

\Subsection{Calling conventions}{ghc-calling-conventions}

\Subsubsection{The call/return registers}{ghc-regs}

One of the problems in designing a virtual machine is that we want it
abstract away from tedious machine details but still reveal enough of
the underlying hardware that we can make sensible decisions about code
generation.  A major problem area is the use of registers in
call/return conventions.  On a machine with lots of registers, it's
cheaper to pass arguments and results in registers than to pass them
on the stack.  On a machine with very few registers, it's cheaper to
pass arguments and results on the stack than to use ``virtual
registers'' in memory.  We therefore use a hybrid system: the first
$n$ arguments or results are passed in registers; and the remaining
arguments or results are passed on the stack.  For register-poor
architectures, it is important that we allow $n=0$.

We'll label the arguments and results \Arg{1} \ldots \Arg{m} --- with
the understanding that \Arg{1} \ldots \Arg{n} are in registers and
\Arg{n+1} \ldots \Arg{m} are on top of the stack.

Note that the mapping of arguments \Arg{1} \ldots \Arg{n} to machine
registers depends on the \emph{kinds} of the arguments.  For example,
if the first argument is a Float, we might pass it in a different
register from if it is an Int.  In fact, we might find that a given
architecture lets us pass varying numbers of arguments according to
their types.  For example, if a CPU has 2 Int registers and 2 Float
registers then we could pass between 2 and 4 arguments in machine
registers --- depending on whether they all have the same kind or they
have different kinds.

\Subsubsection{Entering closures}{entering-closures}

To evaluate a closure we jump to the entry code for the closure
passing a pointer to the closure in \Arg{1} so that the entry code can
access its environment.

\Subsubsection{Function call}{ghc-fun-call}

The function-call mechanism is obviously crucial.  There are five different
cases to consider:
\begin{enumerate}

\item \emph{Known combinator (function with no free variables) and
enough arguments.}

A fast call can be made: push excess arguments onto stack and jump to
function's \emph{fast entry point} passing arguments in \Arg{1} \ldots
\Arg{m}.

The \emph{fast entry point} is only called with exactly the right
number of arguments (in \Arg{1} \ldots \Arg{m}) so it can instantly
start doing useful work without first testing whether it has enough
registers or having to pop them off the stack first.

\item \emph{Known combinator and insufficient arguments.}

A slow call can be made: push all arguments onto stack and jump to
function's \emph{slow entry point}.

Any unpointed arguments which are pushed on the stack must be tagged.
This means pushing an extra word on the stack below the unpointed
words, containing the number of unpointed words above it.

%Todo: forward ref about tagging?
%Todo: picture?

The \emph{slow entry point} might be called with insufficient arguments
and so it must test whether there are enough arguments on the stack.
This \emph{argument satisfaction check} consists of checking that
@Su-Sp@ is big enough to hold all the arguments (including any tags).

\begin{itemize} 

\item If the argument satisfaction check fails, it is because there is
one or more update frames on the stack before the rest of the
arguments that the function needs.  In this case, we construct a PAP
(partial application, \secref{PAP}) containing the arguments
which are on the stack.  The PAP construction code will return to the
update frame with the address of the PAP in \Arg{1}.

\item If the argument satisfaction check succeeds, we jump to the fast
entry point with the arguments in \Arg{1} \ldots \Arg{arity}.

If the fast entry point expects to receive some of \Arg{i} on the
stack, we can reduce the amount of movement required by making the
stack layout for the fast entry point look like the stack layout for
the slow entry point.  Since the slow entry point is entered with the
first argument on the top of the stack and with tags in front of any
unpointed arguments, this means that if \Arg{i} is unpointed, there
should be space below it for a tag and that the highest numbered
argument should be passed on the top of the stack.

We usually arrange that the fast entry point is placed immediately
after the slow entry point --- so we can just ``fall through'' to the
fast entry point without performing a jump.

\end{itemize}


\item \emph{Known function closure (function with free variables) and
enough arguments.}

A fast call can be made: push excess arguments onto stack and jump to
function's \emph{fast entry point} passing a pointer to closure in
\Arg{1} and arguments in \Arg{2} \ldots \Arg{m+1}.

Like the fast entry point for a combinator, the fast entry point for a
closure is only called with appropriate values in \Arg{1} \ldots
\Arg{m+1} so we can start work straight away.  The pointer to the
closure is used to access the free variables of the closure.


\item \emph{Known function closure and insufficient arguments.}

A slow call can be made: push all arguments onto stack and jump to the
closure's slow entry point passing a pointer to the closure in \Arg{1}.

Again, the slow entry point performs an argument satisfaction check
and either builds a PAP or pops the arguments off the stack into
\Arg{2} \ldots \Arg{m+1} and jumps to the fast entry point.


\item \emph{Unknown function closure, thunk or constructor.}

Sometimes, the function being called is not statically identifiable.
Consider, for example, the @compose@ function:
\begin{verbatim}
  compose f g x = f (g x)
\end{verbatim}
Since @f@ and @g@ are passed as arguments to @compose@, the latter has
to make a heap call.  In a heap call the arguments are pushed onto the
stack, and the closure bound to the function is entered.  In the
example, a thunk for @(g x)@ will be allocated, (a pointer to it)
pushed on the stack, and the closure bound to @f@ will be
entered. That is, we will jump to @f@s entry point passing @f@ in
\Arg{1}.  If \Arg{1} is passed on the stack, it is pushed on top of
the thunk for @(g x)@.

The \emph{entry code} for an updateable thunk (which must have arity 0)
pushes an update frame on the stack and starts executing the body of
the closure --- using \Arg{1} to access any free variables.  This is
described in more detail in \secref{data-updates}.

The \emph{entry code} for a non-updateable closure is just the
closure's slow entry point.

\end{enumerate}

In addition to the above considerations, if there are \emph{too many}
arguments then the extra arguments are simply pushed on the stack with
appropriate tags.

To summarise, a closure's standard (slow) entry point performs the
following:

\begin{description}
\item[Argument satisfaction check.] (function closure only)
\item[Stack overflow check.]
\item[Heap overflow check.]
\item[Copy free variables out of closure.] %Todo: why?
\item[Eager black holing.] (updateable thunk only) %Todo: forward ref.
\item[Push update frame.]
\item[Evaluate body of closure.]
\end{description}


\Subsection{Case expressions and return conventions}{return-conventions}

The \emph{evaluation} of a thunk is always initiated by
a @case@ expression.  For example:
\begin{verbatim}
  case x of (a,b) -> E
\end{verbatim}

The code for a @case@ expression looks like this:

\begin{itemize}
\item Push the free variables of the branches on the stack (fv(@E@) in
this case).
\item  Push a \emph{return address} on the stack.
\item  Evaluate the scrutinee (@x@ in this case).
\end{itemize}

Once evaluation of the scrutinee is complete, execution resumes at the
return address, which points to the code for the expression @E@.

When execution resumes at the return point, there must be some {\em
return convention} that defines where the components of the pair, @a@
and @b@, can be found.  The return convention varies according to the
type of the scrutinee @x@:

\begin{itemize}

\item 

(A space for) the return address is left on the top of the stack.
Leaving the return address on the stack ensures that the top of the
stack contains a valid activation record
(\secref{activation-records}) --- should a garbage
collection be required.

\item If @x@ has a boxed type (e.g.~a data constructor or a function),
a pointer to @x@ is returned in \Arg{1}.

\ToDo{Warn that components of E should be extracted as soon as
possible to avoid a space leak.}

\item If @x@ is an unboxed type (e.g.~@Int#@ or @Float#@), @x@ is
returned in \Arg{1}

\item If @x@ is an unboxed tuple constructor, the components of @x@
are returned in \Arg{1} \ldots \Arg{n} but no object is constructed in
the heap.  

When passing an unboxed tuple to a function, the components are
flattened out and passed in \Arg{1} \ldots \Arg{n} as usual.

\end{itemize}

\Subsection{Vectored Returns}{vectored-returns}

Many algebraic data types have more than one constructor.  For
example, the @Maybe@ type is defined like this:
\begin{verbatim}
  data Maybe a = Nothing | Just a
\end{verbatim}
How does the return convention encode which of the two constructors is
being returned?  A @case@ expression scrutinising a value of @Maybe@
type would look like this: 
\begin{verbatim}
  case E of 
    Nothing -> ...
    Just a  -> ...
\end{verbatim}
Rather than pushing a return address before evaluating the scrutinee,
@E@, the @case@ expression pushes (a pointer to) a \emph{return
vector}, a static table consisting of two code pointers: one for the
@Just@ alternative, and one for the @Nothing@ alternative.  

\begin{itemize}

\item

The constructor @Nothing@ returns by jumping to the first item in the
return vector with a pointer to a (statically built) Nothing closure
in \Arg{1}.  

It might seem that we could avoid loading \Arg{1} in this case since the
first item in the return vector will know that @Nothing@ was returned
(and can easily access the Nothing closure in the (unlikely) event
that it needs it.  The only reason we load \Arg{1} is in case we have to
perform an update (\secref{data-updates}).

\item 

The constructor @Just@ returns by jumping to the second element of the
return vector with a pointer to the closure in \Arg{1}.  

\end{itemize}

In this way no test need be made to see which constructor returns;
instead, execution resumes immediately in the appropriate branch of
the @case@.

\Subsection{Direct Returns}{direct-returns}

When a datatype has a large number of constructors, it may be
inappropriate to use vectored returns.  The vector tables may be
large and sparse, and it may be better to identify the constructor
using a test-and-branch sequence on the tag.  For this reason, we
provide an alternative return convention, called a \emph{direct
return}.

In a direct return, the return address pushed on the stack really is a
code pointer.  The returning code loads a pointer to the closure being
returned in \Arg{1} as usual, and also loads the tag into \Arg{2}.
The code at the return address will test the tag and jump to the
appropriate code for the case branch.  If \Arg{2} isn't mapped to a
real machine register on this architecture, then we don't load it on a
return, instead using the tag directly from the info table.

The choice of whether to use a vectored return or a direct return is
made on a type-by-type basis --- up to a certain maximum number of
constructors imposed by the update mechanism
(\secref{data-updates}).

Single-constructor data types also use direct returns, although in
that case there is no need to return a tag in \Arg{2}.

\ToDo{for a nullary constructor we needn't return a pointer to the
constructor in \Arg{1}.}

\Subsection{Updates}{data-updates}

The entry code for an updatable thunk (which must be of arity 0):

\begin{itemize}
\item copies the free variables out of the thunk into registers or
  onto the stack.
\item pushes an \emph{update frame} onto the stack.

An update frame is a small activation record consisting of
\begin{center}
\begin{tabular}{|l|l|l|}
\hline
\emph{Fixed header} & \emph{Update Frame link} & \emph{Updatee} \\
\hline
\end{tabular}
\end{center}

\note{In the semantics part of the STG paper (section 5.6), an update
frame consists of everything down to the last update frame on the
stack.  This would make sense too --- and would fit in nicely with
what we're going to do when we add support for speculative
evaluation.}
\ToDo{I think update frames contain cost centres sometimes}

\item If we are doing ``eager blackholing,'' we then overwrite the
thunk with a black hole (\secref{BLACKHOLE}).  Otherwise, we leave it
to the garbage collector to black hole the thunk.

\item 
Start evaluating the body of the expression.

\end{itemize}

When the expression finishes evaluation, it will enter the update
frame on the top of the stack.  Since the returner doesn't know
whether it is entering a normal return address/vector or an update
frame, we follow exactly the same conventions as return addresses and
return vectors.  That is, on entering the update frame:

\begin{itemize} 
\item The value of the thunk is in \Arg{1}.  (Recall that only thunks
are updateable and that thunks return just one value.)

\item If the data type is a direct-return type rather than a
vectored-return type, then the tag is in \Arg{2}.

\item The update frame is still on the stack.
\end{itemize}

We can safely share a single statically-compiled update function
between all types.  However, the code must be able to handle both
vectored and direct-return datatypes.  This is done by arranging that
the update code looks like this:

\begin{verbatim}
                |       ^       |
                | return vector |
                |---------------|
                |  fixed-size   |
                |  info table   |
                |---------------|  <- update code pointer
                |  update code  |
                |       v       |
\end{verbatim}

Each entry in the return vector (which is large enough to cover the
largest vectored-return type) points to the update code.

The update code:
\begin{itemize}
\item overwrites the \emph{updatee} with an indirection to \Arg{1};
\item loads @Su@ from the Update Frame link;
\item removes the update frame from the stack; and 
\item enters \Arg{1}.
\end{itemize}

We enter \Arg{1} again, having probably just come from there, because
it knows whether to perform a direct or vectored return.  This could
be optimised by compiling special update code for each slot in the
return vector, which performs the correct return.

\Subsection{Semi-tagging}{semi-tagging}

When a @case@ expression evaluates a variable that might be bound
to a thunk it is often the case that the scrutinee is already evaluated.
In this case we have paid the penalty of (a) pushing the return address (or
return vector address) on the stack, (b) jumping through the info pointer
of the scrutinee, and (c) returning by an indirect jump through the
return address on the stack.

If we knew that the scrutinee was already evaluated we could generate
(better) code which simply jumps to the appropriate branch of the
@case@ with a pointer to the scrutinee in \Arg{1}.  (For direct
returns to multiconstructor datatypes, we might also load the tag into
\Arg{2}).

An obvious idea, therefore, is to test dynamically whether the heap
closure is a value (using the tag in the info table).  If not, we
enter the closure as usual; if so, we jump straight to the appropriate
alternative.  Here, for example, is pseudo-code for the expression
@(case x of { (a,_,c) -> E }@:
\begin{verbatim}
      \Arg{1} = <pointer to x>;
      tag = \Arg{1}->entry->tag;
      if (isWHNF(tag)) {
          Sp--;  \\ insert space for return address
          goto ret;
      }
      push(ret);           
      goto \Arg{1}->entry;
      
      <info table for return address goes here>
ret:  a = \Arg{1}->data1; \\ suck out a and c to avoid space leak
      c = \Arg{1}->data3;
      <code for E2>
\end{verbatim}
and here is the code for the expression @(case x of { [] -> E1; x:xs -> E2 }@:
\begin{verbatim}
      \Arg{1} = <pointer to x>;
      tag = \Arg{1}->entry->tag;
      if (isWHNF(tag)) {
          Sp--;  \\ insert space for return address
          goto retvec[tag];
      }
      push(retinfo);          
      goto \Arg{1}->entry;
      
      .addr ret2
      .addr ret1
retvec:           \\ reversed return vector
      <return info table for case goes here>
retinfo:
      panic("Direct return into vectored case");
      
ret1: <code for E1>

ret2: x  = \Arg{1}->head;
      xs = \Arg{1}->tail;
      <code for E2>
\end{verbatim}
There is an obvious cost in compiled code size (but none in the size
of the bytecodes).  There is also a cost in execution time if we enter
more thunks than data constructors.

Both the direct and vectored returns are easily modified to chase chains
of indirections too.  In the vectored case, this is most easily done by
making sure that @IND = TAG_1 - 1@, and adding an extra field to every
return vector.  In the above example, the indirection code would be
\begin{verbatim}
ind:  \Arg{1} = \Arg{1}->next;
      goto ind_loop;
\end{verbatim}
where @ind_loop@ is the second line of code.

Note that we have to leave space for a return address since the return
address expects to find one.  If the body of the expression requires a
heap check, we will actually have to write the return address before
entering the garbage collector.


\Subsection{Heap and Stack Checks}{heap-and-stack-checks}

The storage manager detects that it needs to garbage collect the old
generation when the evaluator requests a garbage collection without
having moved the heap pointer since the last garbage collection.  It
is therefore important that the GC routines \emph{not} move the heap
pointer unless the heap check fails.  This is different from what
happens in the current STG implementation.

Assuming that the stack can never shrink, we perform a stack check
when we enter a closure but not when we return to a return
continuation.  This doesn't work for heap checks because we cannot
predict what will happen to the heap if we call a function.

If we wish to allow the stack to shrink, we need to perform a stack
check whenever we enter a return continuation.  Most of these checks
could be eliminated if the storage manager guaranteed that a stack
would always have 1000 words (say) of space after it was shrunk.  Then
we can omit stack checks for less than 1000 words in return
continuations.

When an argument satisfaction check fails, we need to push the closure
(in R1) onto the stack - so we need to perform a stack check.  The
problem is that the argument satisfaction check occurs \emph{before}
the stack check.  The solution is that the caller of a slow entry
point or closure will guarantee that there is at least one word free
on the stack for the callee to use.  

Similarily, if a heap or stack check fails, we need to push the arguments
and closure onto the stack.  If we just came from the slow entry point, 
there's certainly enough space and it is the responsibility of anyone
using the fast entry point to guarantee that there is enough space.

\ToDo{Be more precise about how much space is required - document it
in the calling convention section.}

\Subsection{Handling interrupts/signals}{signals}

\begin{verbatim}
May have to keep C stack pointer in register to placate OS?
May have to revert black holes - ouch!
\end{verbatim}



\section{The Loader}
\section{The Compilers}

\iffalse
\part{Old stuff - needs to be mined for useful info}

\section{The Scheduler}

The Scheduler is the heart of the run-time system.  A running program
consists of a single running thread, and a list of runnable and
blocked threads.  The running thread returns to the scheduler when any
of the following conditions arises:

\begin{itemize}
\item A heap check fails, and a garbage collection is required
\item Compiled code needs to switch to interpreted code, and vice
versa.
\item The thread becomes blocked.
\item The thread is preempted.
\end{itemize}

A running system has a global state, consisting of

\begin{itemize}
\item @Hp@, the current heap pointer, which points to the next
available address in the Heap.
\item @HpLim@, the heap limit pointer, which points to the end of the
heap.
\item The Thread Preemption Flag, which is set whenever the currently
running thread should be preempted at the next opportunity.
\item A list of runnable threads. 
\item A list of blocked threads.
\end{itemize}

Each thread is represented by a Thread State Object (TSO), which is
described in detail in \secref{TSO}.

The following is pseudo-code for the inner loop of the scheduler
itself.

\begin{verbatim}
while (threads_exist) {
  // handle global problems: GC, parallelism, etc
  if (need_gc) gc();  
  if (external_message) service_message();
  // deal with other urgent stuff

  pick a runnable thread;
  do {
    // enter object on top of stack
    // if the top object is a BCO, we must enter it
    // otherwise apply any heuristic we wish.
    if (thread->stack[thread->sp]->info.type == BCO) {
	status = runHugs(thread,&smInfo);
    } else {
	status = runGHC(thread,&smInfo);
    }
    switch (status) {  // handle local problems
      case (StackOverflow): enlargeStack; break;
      case (Error e)      : error(thread,e); break;
      case (ExitWith e)   : exit(e); break;
      case (Yield)        : break;
    }
  } while (thread_runnable);
}
\end{verbatim}

\Subsection{Invoking the garbage collector}{ghc-invoking-gc}

\Subsection{Putting the thread to sleep}{ghc-thread-sleeps}

\Subsection{Calling C from Haskell}{ghc-ccall}

We distinguish between "safe calls" where the programmer guarantees
that the C function will not call a Haskell function or, in a
multithreaded system, block for a long period of time and "unsafe
calls" where the programmer cannot make that guarantee.  

Safe calls are performed without returning to the scheduler and are
discussed elsewhere (\ToDo{discuss elsewhere}).

Unsafe calls are performed by returning an array (outside the Haskell
heap) of arguments and a C function pointer to the scheduler.  The
scheduler allocates a new thread from the operating system
(multithreaded system only), spawns a call to the function and
continues executing another thread.  When the ccall completes, the
thread informs the scheduler and the scheduler adds the thread to the
runnable threads list.  

\ToDo{Describe this in more detail.}


\Subsection{Calling Haskell from C}{ghc-c-calls-haskell}

When C calls a Haskell closure, it sends a message to the scheduler
thread.  On receiving the message, the scheduler creates a new Haskell
thread, pushes the arguments to the C function onto the thread's stack
(with tags for unboxed arguments) pushes the Haskell closure and adds
the thread to the runnable list so that it can be entered in the
normal way.

When the closure returns, the scheduler sends back a message which
awakens the (C) thread.  

\ToDo{Do we need to worry about the garbage collector deallocating the
thread if it gets blocked?}

\Subsection{Switching Worlds}{switching-worlds}

\ToDo{This has all changed: we always leave a closure on top of the
stack if we mean to continue executing it.  The scheduler examines the
top of the stack and tries to guess which world we want to be in.  If
it finds a @BCO@, it certainly enters Hugs, if it finds a @GHC@
closure, it certainly enters GHC and if it finds a standard closure,
it is free to choose either one but it's probably best to enter GHC
for everything except @BCO@s and perhaps @AP@s.}

Because this is a combined compiled/interpreted system, the
interpreter will sometimes encounter compiled code, and vice-versa.

All world-switches go via the scheduler, ensuring that the world is in
a known state ready to enter either compiled code or the interpreter.
When a thread is run from the scheduler, the @whatNext@ field in the
TSO (\secref{TSO}) is checked to find out how to execute the
thread.

\begin{itemize}
\item If @whatNext@ is set to @ReturnGHC@, we load up the required
registers from the TSO and jump to the address at the top of the user
stack.
\item If @whatNext@ is set to @EnterGHC@, we load up the required
registers from the TSO and enter the closure pointed to by the top
word of the stack.
\item If @whatNext@ is set to @EnterHugs@, we enter the top thing on
the stack, using the interpreter.
\end{itemize}

There are four cases we need to consider:

\begin{enumerate}
\item A GHC thread enters a Hugs-built closure.
\item A GHC thread returns to a Hugs-compiled return address.
\item A Hugs thread enters a GHC-built closure.
\item A Hugs thread returns to a Hugs-compiled return address.
\end{enumerate}

GHC-compiled modules cannot call functions in a Hugs-compiled module
directly, because the compiler has no information about arities in the
external module.  Therefore it must assume any top-level objects are
CAFs, and enter their closures.

\ToDo{Hugs-built constructors?}

We now examine the various cases one by one and describe how the
switch happens in each situation.

\subsection{A GHC thread enters a Hugs-built closure}
\label{sec:ghc-to-hugs-switch}

There is three possibilities: GHC has entered a @PAP@, or it has
entered a @AP@, or it has entered the BCO directly (for a top-level
function closure).  @AP@s and @PAP@s are ``standard closures'' and
so do not require us to enter the bytecode interpreter.

The entry code for a BCO does the following:

\begin{itemize}
\item Push the address of the object entered on the stack.
\item Save the current state of the thread in its TSO.
\item Return to the scheduler, setting @whatNext@ to @EnterHugs@.
\end{itemize}

BCO's for thunks and functions have the same entry conventions as
slow entry points: they expect to find their arguments on the stac
with unboxed arguments preceded by appropriate tags.

\subsection{A GHC thread returns to a Hugs-compiled return address}
\label{sec:ghc-to-hugs-switch}

Hugs return addresses are laid out as in \figref{hugs-return-stack}.
If GHC is returning, it will return to the address at the top of the
stack, namely @HUGS_RET@.  The code at @HUGS_RET@ performs the
following:

\begin{itemize}
\item pushes \Arg{1} (the return value) on the stack.
\item saves the thread state in the TSO
\item returns to the scheduler with @whatNext@ set to @EnterHugs@.
\end{itemize}

\noindent When Hugs runs, it will enter the return value, which will
return using the correct Hugs convention
(\secref{hugs-return-convention}) to the return address underneath it
on the stack.

\subsection{A Hugs thread enters a GHC-compiled closure}
\label{sec:hugs-to-ghc-switch}

Hugs can recognise a GHC-built closure as not being one of the
following types of object:

\begin{itemize}
\item A @BCO@,
\item A @AP@,
\item A @PAP@,
\item An indirection, or
\item A constructor.
\end{itemize}

When Hugs is called on to enter a GHC closure, it executes the
following sequence of instructions:

\begin{itemize}
\item Push the address of the closure on the stack.
\item Save the current state of the thread in the TSO.
\item Return to the scheduler, with the @whatNext@ field set to
@EnterGHC@.
\end{itemize}

\subsection{A Hugs thread returns to a GHC-compiled return address}
\label{sec:hugs-to-ghc-switch}

When Hugs encounters a return address on the stack that is not
@HUGS_RET@, it knows that a world-switch is required.  At this point
the stack contains a pointer to the return value, followed by the GHC
return address.  The following sequence is then performed:

\begin{itemize}
\item save the state of the thread in the TSO.
\item return to the scheduler, setting @whatNext@ to @EnterGHC@.
\end{itemize}

The first thing that GHC will do is enter the object on the top of the
stack, which is a pointer to the return value.  This value will then
return itself to the return address using the GHC return convention.


\fi


\part{History}

We're nuking the following:

\begin{itemize}
\item
  Two stacks

\item
  Return in registers.
  This lets us remove update code pointers from info tables,
  removes the need for phantom info tables, simplifies 
  semi-tagging, etc.

\item
  Threaded GC.
  Careful analysis suggests that it doesn't buy us very much
  and it is hard to work with.

  Eliminating threaded GCs eliminates the desire to share SMReps
  so they are (once more) part of the Info table.

\item
  RetReg.
  Doesn't buy us anything on a register-poor architecture and
  isn't so important if we have semi-tagging.

\begin{verbatim}
    - Probably bad on register poor architecture 
    - Can avoid need to write return address to stack on reg rich arch.
      - when a function does a small amount of work, doesn't 
  	enter any other thunks and then returns.
  	eg entering a known constructor (but semitagging will catch this)
    - Adds complications
\end{verbatim}

\item
  Update in place

  This lets us drop CONST closures and CHARLIKE closures (assuming we
  don't support Unicode).  The only point of these closures was to 
  avoid updating with an indirection.

  We also drop @MIN_UPD_SIZE@ --- all we need is space to insert an
  indirection or a black hole.

\item
  STATIC SMReps are now called CONST

\item
  @MUTVAR@ is new

\item The profiling ``kind'' field is now encoded in the @INFO_TYPE@ field.
This identifies the general sort of the closure for profiling purposes.

\item Various papers describe deleting update frames for unreachable objects.
  This has never been implemented and we don't plan to anytime soon.

\end{itemize}


\end{document}


