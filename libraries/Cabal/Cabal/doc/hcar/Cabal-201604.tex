\documentclass[DIV16,twocolumn,10pt]{scrreprt}
\usepackage{paralist}
\usepackage{graphicx}
\usepackage[final]{hcar}

%include polycode.fmt

\begin{document}

\begin{hcarentry}{Cabal}
\report{Mikhail Glushenkov}
\status{Active}
\participants{\href{https://github.com/haskell/cabal/graphs/contributors}{Cabal contributors}}% optional
\makeheader

\subsubsection*{Background}

Cabal is the standard packaging system for Haskell software. It specifies a
standard way in which Haskell libraries and applications can be packaged so that
it is easy for consumers to use them, or re-package them, regardless of the
Haskell implementation or installation platform.

\texttt{cabal-install} is the command line interface for the Cabal and Hackage
system. It provides a command line program \texttt{cabal} which has sub-commands
for installing and managing Haskell packages.

\subsubsection*{Recent Progress}

We've just released versions 1.24 of Cabal and \texttt{cabal-install}. 1.24
incorporates more than a thousand commits by
\href{https://gist.github.com/23Skidoo/62544d7e0352037749eec7344788831c}{89
  different contributors}. Main user-visible changes in this release are:

\begin{itemize}
\item Nix-style local builds in \texttt{cabal-install} (so far only a technical
  preview). See
  \href{http://blog.ezyang.com/2016/05/announcing-cabal-new-build-nix-style-local-builds/}{this
    post} by Edward Z. Yang for more details.
\item Integration of a new security scheme for Hackage based on
  \href{https://theupdateframework.github.io/}{The Update Framework}. So far
  this is not enabled by default, pending some changes on the Hackage side. See
  \href{http://www.well-typed.com/blog/2015/08/hackage-security-beta/}{these}
  \href{http://www.well-typed.com/blog/2015/07/hackage-security-alpha/}{three}
  \href{http://www.well-typed.com/blog/2015/04/improving-hackage-security/}{posts}
  by Edsko de Vries and Duncan Coutts for more information.
\item Support for specifying setup script dependencies in \texttt{.cabal}
  files. See
  \href{http://www.well-typed.com/blog/2015/07/cabal-setup-deps/}{this post by
    Duncan Coutts} for more information.
\item Support for HTTPS downloads in \texttt{cabal-install}. HTTPS is now used
  by default for downloads from Hackage.
\item \texttt{cabal upload} learned how to upload documentation to Hackage
  (\texttt{cabal upload --doc}).
\item In related news, \texttt{cabal haddock} now can generate documentation
  intended for uploading to Hackage (\texttt{cabal haddock
    --for-hackage}). \texttt{cabal upload --doc} runs this command automatically
  if the documentation for current package wasn't generated yet.
\item New \texttt{cabal-install} command: \texttt{gen-bounds}. See
  \href{http://softwaresimply.blogspot.se/2015/08/cabal-gen-bounds-easy-generation-of.html}{here}
  for more information.
\item It's now possible to limit the scope of \texttt{--allow-newer} to single
  packages in the install plan, both on the command line and in the config
  file. See \href{https://github.com/haskell/cabal/issues/2756}{here} for an
  example.
\item New \texttt{cabal user-config} subcommand: \texttt{init}, which creates a
  default \texttt{\textasciitilde{}/.cabal/config} file.
\item New config file field: \texttt{extra-framework-dirs} (extra locations to
  find OS X frameworks in).
\item \texttt{cabal-install} solver
  \href{https://github.com/haskell/cabal/pull/2873}{now takes information about
    extensions and language flavours into account}.
\item New \texttt{cabal-install} option:
  \href{https://github.com/haskell/cabal/pull/2578}{\texttt{--offline}}, which
  prevents \texttt{cabal-install} from downloading anything from the Internet.
\item New \texttt{cabal upload} option
  \href{https://github.com/haskell/cabal/pull/2506}{\texttt{-P}/\texttt{--password-command}}
  for reading Hackage password from arbitrary program output.
\item Support for GHC 8 (NB: old versions of Cabal won't work with this version
  of GHC).
\end{itemize}

Full list of changes in Cabal 1.24 is available
\href{http://hackage.haskell.org/package/Cabal-1.24.0.0/changelog}{here}; full
list of changes in \texttt{cabal-install} 1.24 is available
\href{http://hackage.haskell.org/package/cabal-install-1.24.0.0/changelog}{here}.

\subsubsection*{Looking Forward}

We plan to make a new release of Cabal/\texttt{cabal-install} approximately 6
months after 1.24 -- that is, in late October or early November 2016. Main
features that are currently targeted at 1.26 are:

\begin{itemize}
\item Further work on nix-style local builds, perhaps making that code path the
  default.
\item Enabling Hackage Security by default.
\item Native support for
  \href{https://github.com/haskell/cabal/pull/2540}{``foreign libraries''}:
  Haskell libraries that are intended to be used by non-Haskell code.
\item New Parsec-based parser for \texttt{.cabal} files.
\end{itemize}

We would like to encourage people considering contributing to take a look at
\href{https://github.com/haskell/cabal/issues/}{the bug tracker on GitHub}, take
part in discussions on tickets and pull requests, or submit their own. The bug
tracker is reasonably well maintained and it should be relatively clear to new
contributors what is in need of attention and which tasks are considered
relatively easy. For more in-depth discussion there is also the
\href{https://mail.haskell.org/mailman/listinfo/cabal-devel}{\texttt{cabal-devel}}
mailing list.

\FurtherReading
  Cabal homepage:\hfill\url{https://www.haskell.org/cabal/}\\
  Cabal on GitHub:\hfill\url{https://github.com/haskell/cabal}

\end{hcarentry}

\end{document}
